% --------- deployment guide for the report generator ---------		
	\subsection[file config]{Umgebung Konfigurieren}
	\subsubsection{1 - Mit Setup Script}
	
	\lstset{style=commands}
	\begin{lstlisting}[caption={Ausführen des 'install.sh' Scripts.}]
		elbrus@server:~/var/elbrus$ bash report-generator/install.sh
		Do you want to proceed with setup of the 'report-generator'? (y/n) y
		
		Where is the shared config stored [/var/elbrus/shared/.config]:
		
		Is the shared config stored at '/var/elbrus/shared/.config' ? (y/n/exit) y
		Install dependencies ...
		
		...
		
		elbrus@server:~/var/elbrus$
	\end{lstlisting}
	
	\subsubsection{2 - Ohne Setup Script}
	
	\lstset{style=files}
	\begin{lstlisting}[caption={Anhand von '.env.example' eigene '.env' Datei anlegen.}, language=bash]
		#global
		SHAREDCONFIG=/var/elbrus/shared/.config
	\end{lstlisting}
	
	\lstset{style=commands}
	\begin{lstlisting}[caption={Installieren von fehlenden python3 Packages.}]
		elbrus@server:~/var/elbrus$ pip3 install -r \
		report-generator/requirements.txt
	\end{lstlisting}
	\newpage
	
	\subsection[systemd service]{Der Systemd Service}
	
	\lstset{style=files}
	\begin{lstlisting}[caption={snmp-manager.service.example - Die Variable 'WorkingDirectory' sowie die Variable 'User' anpassen.},language=bash ,keywords={WorkingDirectory, User}, keywordstyle=\color{red}, firstnumber=5]
		...
		#job is starting immediatly after the start action has been called
		Type=simple
		#the user to execute the script
		User=elbrus
		#the working directory
		WorkingDirectory=/var/elbrus/report-generator/src/
		#which script should be executed
		ExecStart=python3 main.py
		...
	\end{lstlisting}
	
	\lstset{style=commands}
	\begin{lstlisting}[caption={Kopieren des Serviceprogrammes.}]
		elbrus@server:~/var/elbrus$ sudo cp report-generator/src/report-generator.service\
		.example /etc/systemd/system/report-generator.service
	\end{lstlisting}
	
	\lstset{style=commands}
	\begin{lstlisting}[caption={Kopieren des Zeitplanungsprogrammes.}]
		elbrus@server:~/var/elbrus$ sudo cp snmp-manager/src/report-generator-schedule.timer\
		.example /etc/systemd/system/report-generator-schedule.timer
	\end{lstlisting}
	
	\lstset{style=commands}
	\begin{lstlisting}[caption={Neuladen des 'systemctl' Deamons.}]
		elbrus@server:~/var/elbrus$ sudo systemctl daemon-reload
	\end{lstlisting}
	
	\lstset{style=commands}
	\begin{lstlisting}[caption={Aktivieren des Serviceprogrammes.}]
		elbrus@server:~/var/elbrus$ sudo systemctl enable report-generator.service
	\end{lstlisting}
	
	\lstset{style=commands}
	\begin{lstlisting}[caption={Aktivieren des Zeitplanungsprogrammes.}]
		elbrus@server:~/var/elbrus$ sudo systemctl enable report-generator-schedule.timer
	\end{lstlisting}
	
	\lstset{style=commands}
	\begin{lstlisting}[caption={Starten des Zeitplanungsprogrammes.}]
		elbrus@server:~/var/elbrus$ sudo systemctl start report-generator-schedule.timer
	\end{lstlisting}