% --------- deployment guide for the uptime monitor ---------	
	\subsection[file config]{Umgebung Konfigurieren}
	\subsubsection{1 - Mit Setup Script}
	
	\lstset{style=commands}
	\begin{lstlisting}[caption={Ausführen des 'install.sh' Scripts.}]
		elbrus@server:~$ cd /var/elbrus
		elbrus@server:~$ bash uptime_monitor/install.sh
		Do you want to proceed with the setup of the 'uptime_monitor'? (y/n) y
		we will proceed
		
		Where is the shared config stored [/var/elbrus/shared/.config]
		
		Is the shared config stored at '/var/elbrus/shared/.config'? (y/n/exit) y
		we will proceed
		
		#global
		SHAREDCONFIG=/var/elbrus/shared/.config
		
		#config
		INITIALPING=1
		STATISTICPING=10
		Cleaning up...
		elbrus@server:~$
	\end{lstlisting}
	
	\subsubsection{2 - Ohne Setup Script}
	
	\lstset{style=files}
	\begin{lstlisting}[caption={Anhand von '.env.example' eigene '.env' Datei anlegen.}, language=bash]
		#global
		SHAREDCONFIG=/var/elbrus/shared/.config
		
		#config
		# Initial pings to see if device is alive
		INITIALPING=1
		# Pings to get the availability statistic
		STATISTICPING=10
	\end{lstlisting}
	\newpage
	
	\subsection[systemd service]{Der Systemd Service}
	
	\lstset{style=files}
	\begin{lstlisting}[caption={uptime\_monitor.service.example - Die Variable 'WorkingDirectory' sowie die Variable 'User' anpassen.},language=bash ,keywords={WorkingDirectory, User}, keywordstyle=\color{red}, firstnumber=5]
		...
		#job is starting immediatly after the start action has been called
		Type=simple
		#the user to execute the script
		User=elbrus
		#the working directory
		WorkingDirectory=/var/elbrus/uptime_monitor
		#which script should be executed
		ExecStart=/bin/bash uptime_monitor.sh
		...
	\end{lstlisting}
	
	\lstset{style=commands}
	\begin{lstlisting}[caption={Kopieren des Serviceprogrammes.}]
		elbrus@server:/var/elbrus$ sudo cp uptime_monitor/uptime_monitor.service.example \
		/etc/systemd/system/uptime_monitor.service
	\end{lstlisting}
	
	\lstset{style=commands}
	\begin{lstlisting}[caption={Kopieren des Zeitplanungsprogrammes.}]
		elbrus@server:/var/elbrus$ sudo cp uptime_monitor/uptime_monitor-schedule.timer.example \
		/etc/systemd/system/uptime_monitor-schedule.timer
	\end{lstlisting}
	
	\lstset{style=commands}
	\begin{lstlisting}[caption={Neuladen des 'systemctl' Deamons.}]
		elbrus@server:/var/elbrus$ sudo systemctl daemon-reload
	\end{lstlisting}
	
	\lstset{style=commands}
	\begin{lstlisting}[caption={Aktivieren des Serviceprogrammes.}]
		elbrus@server:/var/elbrus$ sudo systemctl enable uptime_monitor.service
	\end{lstlisting}
	
	\lstset{style=commands}
	\begin{lstlisting}[caption={Aktivieren des Zeitplanungsprogrammes.}]
		elbrus@server:/var/elbrus$ sudo systemctl enable uptime_monitor-schedule.timer
	\end{lstlisting}
	
	\lstset{style=commands}
	\begin{lstlisting}[caption={Starten des Zeitplanungsprogrammes.}]
		elbrus@server:/var/elbrus$ sudo systemctl start uptime_monitor-schedule.timer
	\end{lstlisting}