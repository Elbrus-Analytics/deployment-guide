% --------- deployment guide for the office 365 get endpoint ---------
	\subsection[file config]{Umgebung Konfigurieren}
	\subsubsection{1 - Mit Setup Script}
	
	\lstset{style=commands}
	\begin{lstlisting}[caption={Ausführen des 'install.sh' Scripts.}]
		elbrus@server:~$ cd /var/elbrus
		elbrus@server:~$ sudo bash office365-analyzer/src/install.sh
		Do you want to proceed with setup of the 'office365-analyzer'? (y/n) y
		
		Where should the 'office365-analyzer' be stored (dir) [/var/elbrus/office365-analyzer]:
		Where should the log be stored (dir) [/var/elbrus/shared/log]:
		Where is the shared config stored [/var/elbrus/shared/.config]:
		
		Should the 'office365-analyzer' be stored at '/var/elbrus/office365-analyzer' ?
		Should the log be stored at '/var/elbrus/shared/log' ?
		Is the shared config stored at '/var/elbrus/shared/.config' ? (y/n/exit) y
		
		<@\textcolor{codegreen}{Success!}@> .env file was created
		<@\textcolor{codegreen}{Success!}@> systemd service was automatically deployed,
		<@\textcolor{codegreen}{Success!}@> installed requirements
		elbrus@server:~$	
	\end{lstlisting}
	
	\subsubsection{2 - Ohne Setup Script}
	
	\lstset{style=files}
	\begin{lstlisting}[caption={Anhand von '.env.example' eigene '.env' Datei anlegen.}, language=bash]
		#global
		SHAREDCONFIG=/var/elbrus/shared/.config
		
		#paths
		LOGFILEDIR=/var/elbrus/shared/log
		
		#ms url
		MS_URL=https://endpoints.office.com/endpoints/worldwide?clientrequestid=b10c5ed1-bad1-445f-b386-b919946339a7
	\end{lstlisting}
	\newpage

	\subsection[systemd service]{Der Systemd Service}
	
	\lstset{style=files}
	\begin{lstlisting}[caption={uptime\_monitor.service.example - Die Variable 'WorkingDirectory' sowie die Variable 'User' anpassen.},language=bash ,keywords={WorkingDirectory, User}, keywordstyle=\color{red}, firstnumber=4]
		...
		#job is starting immediatly after the start action has been called
		Type=simple
		#the user to execute the script
		User=elbrus
		#the working directory
		WorkingDirectory=/var/elbrus/office365-analyzer/src
		#which script should be executed
		ExecStart=/bin/bash elb-office365-get-endpoints.sh
		...
	\end{lstlisting}
	
	\lstset{style=commands}
	\begin{lstlisting}[caption={Kopieren des Serviceprogrammes.}]
		elbrus@server:/var/elbrus$ sudo cp office365-analyzer/src/office365-get-endpoints\
		.service.example /etc/systemd/system/office365-get-endpoints.service
	\end{lstlisting}
	
	\lstset{style=commands}
	\begin{lstlisting}[caption={Kopieren des Zeitplanungsprogrammes.}]
		elbrus@server:/var/elbrus$ sudo cp office365-analyzer/src/office365-get-endpoints\
		-schedule.timer.example /etc/systemd/system/office365-get-endpoints-schedule.timer
	\end{lstlisting}
	
	\lstset{style=commands}
	\begin{lstlisting}[caption={Neuladen des 'systemctl' Deamons.}]
		elbrus@server:/var/elbrus$ sudo systemctl daemon-reload
	\end{lstlisting}
	
	\lstset{style=commands}
	\begin{lstlisting}[caption={Aktivieren des Serviceprogrammes.}]
		elbrus@server:/var/elbrus$ sudo systemctl enable office365-get-endpoints.service
	\end{lstlisting}
	
	\lstset{style=commands}
	\begin{lstlisting}[caption={Aktivieren des Zeitplanungsprogrammes.}]
		elbrus@server:/var/elbrus$ sudo systemctl enable office365-get-endpoints-schedule.timer
	\end{lstlisting}
	
	\lstset{style=commands}
	\begin{lstlisting}[caption={Starten des Serviceprogrammes \& Zeitplanungsprogrammes.}]
		elbrus@server:/var/elbrus$ sudo systemctl start office365-get-endpoints.service
		elbrus@server:/var/elbrus$ sudo systemctl start office365-get-endpoints-schedule.timer
	\end{lstlisting}