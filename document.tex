\documentclass{article}
\usepackage[utf8]{inputenc}

%%FOR CODES
\usepackage{listings,lstautogobble}
\usepackage{xcolor}
\usepackage{caption}

%%COLORES
\definecolor{realylightgray}{gray}{0.95}
\definecolor{RoyalBlue}{cmyk}{1, 0.50, 0, 0}
\definecolor{capri}{rgb}{0.0, 0.75, 1.0}
\definecolor{deepskyblue}{rgb}{0.0, 0.75, 1.0}
\definecolor{cobalt}{rgb}{0.0, 0.28, 0.67}
\definecolor{codegreen}{rgb}{0,0.6,0}
\definecolor{amber}{rgb}{1.0, 0.49, 0.0}
\definecolor{darkorchid}{rgb}{0.6, 0.2, 0.8}

\lstset{
	showstringspaces=false,
	showtabs=false,
	showspaces=false,
	stringstyle=\ttfamily,
	frameround=ffff,
	frame=single,
	framextopmargin=4pt,
	framexbottommargin=4pt,
	rulecolor=\color{darkgray},
	gobble=12, %align text inside box left <-> right
	belowskip=2em,
	keepspaces=true,
	escapeinside={<@}{@>}
}

%%template for commands
\lstdefinestyle{commands}{
	basicstyle=\scriptsize\ttfamily,
	commentstyle=\ttfamily\itshape\color{gray},
	backgroundcolor=\color{realylightgray},
	keywords=[1]{elbrus@server},
	keywordstyle=[1]\color{cobalt},
	numbers=none,
	breaklines=false
}

%%template for files
\lstdefinestyle{files}{
	basicstyle=\ttfamily\footnotesize,
	commentstyle=\color{codegreen},
	backgroundcolor=\color{realylightgray},    
	deletekeywords=[1]{elbrus@server},                                       
	numbers=left,       
	numbersep=5pt,
	breaklines=true,                                                 
}

\captionsetup[lstlisting]{singlelinecheck=false, } %to align caption left
\lstset{style=commands}

\begin{document}
	\title{Elbrus Analytics - Bereitstellungshandbuch}
	\author{Tobias Schmidt}
	\date{\today}
	\maketitle
	\newpage
	
	\section[Server Infrastruktur]{Server Infrastruktur}
	\subsection{Initiale Server Konfiguration}
	
	\begin{lstlisting}[caption={Installieren des DNF Konfigurations Managers, um in Folge das Remote Docker Repository hinzuzufügen.}]
		elbrus@server:~$ dnf install dnf-plugin-config-manager
	\end{lstlisting}

	\begin{lstlisting}[caption={Installieren und aktivieren von Docker.}]
		elbrus@server:~$ sudo dnf config-manager \
		 --add-repo https://download.docker.com/linux/centos/docker-ce.repo
		
		elbrus@server:~$ sudo dnf install docker-ce docker-ce-cli containerd.io
		
		elbrus@server:~$ systemctl enable docker
		
		elbrus@server:~$ systemctl start docker
	\end{lstlisting}

	\begin{lstlisting}[caption={Berechtigt den User Elbrus 'sudo' zu verwenden. Berechtigt den User Elbrus darüber hinaus Docker ohne 'sudo' aufzurufen. Zudem wird dem User ein Heimverzeichnis angelegt, sowie die 'Bash' als standard Konsole gesetzt.}]
		elbrus@server:~$ useradd -s /bin/bash -G docker,wheel -m elbrus
		
		elbrus@server:~$ passwd elbrus
		Changing password for user elbrus.
		New password:
		Retype new password:
		passwd: all authentication tokens updated successfully.
		elbrus@server:~$
	\end{lstlisting}

	\begin{lstlisting}[caption={Setzen der Zeitzone auf 'Europa/Wien'.}]
		elbrus@server:~$ sudo timedatectl set-timezone Europe/Vienna
	\end{lstlisting}

	\begin{lstlisting}[caption={Installieren von dem 'firewalld' Service.}]
		elbrus@server:~$ sudo dnf install firewalld
	\end{lstlisting}

	\newpage
	\subsection{SSH-Keys}
	Weil der pcap-importer und der report-generator auf zwei verschiedenen Server liegen könnten, muss für die Kommunikation zwischen jenen Sever SSH-Funktionieren.
	\newline\newline
	Dieser Schritt kann übersprungen werden wenn alles auf einem Server installiert wird.
	
	\subsubsection{Capture-Server}
	\begin{lstlisting}[caption={Anlegen der SSH-Keys}]
		elbrus@server:~$ mkdir -p /var/elbrus/shared/.ssh/
		elbrus@server:~$ ssh-keygen -t ecdsa -b 256 -f\
		/var/elbrus/shared/.ssh/id_report_generator_connection -N ''
	\end{lstlisting}

	\begin{lstlisting}[caption={Übertragen der SSH-Keys auf den Database-Server.}]
		elbrus@server:~$ ssh-copy-id -i\
		/var/elbrus/shared/.ssh/id_report_generator_connection.pub\
		 elbrus@10.0.76.220
	\end{lstlisting}

	\subsubsection{Database-Server}
	\begin{lstlisting}[caption={Anlegen der SSH-Keys}]
		elbrus@server:~$ mkdir -p /var/elbrus/shared/.ssh/
		elbrus@server:~$ ssh-keygen -t ecdsa -b 256 -f\
		/var/elbrus/shared/.ssh/id_capture_connection -N ''
	\end{lstlisting}
	
	\begin{lstlisting}[caption={Übertragen der SSH-Keys auf den Capture-Server.}]
		elbrus@server:~$ ssh-copy-id -i\
		/var/elbrus/shared/.ssh/id_capture_connection.pub\
		 elbrus@10.0.76.217
	\end{lstlisting}
	
	\subsection{Ablagestruktur}
	\lstset{style=commands}
	\begin{lstlisting}[caption={Anlegen der Ordnerstruktur.}]
		elbrus@server:~$ mkdir sth
	\end{lstlisting}

	\newpage
	\subsection{Python}
	\subsubsection{1 - Automatische Installation}
	\begin{lstlisting}[caption={Kopieren des Github Repositorys 'report-generator'.}]
		elbrus@server:~$ cd /var/elbrus
		elbrus@server:~/var/elbrus$ git https://github.com/\
		Elbrus-Analytics/report-generator.git
	\end{lstlisting}

	\begin{lstlisting}[caption={Ausführen des 'pythonSourceInstall.sh' Scripts.}]
		elbrus@server:~$ bash report-generator/pythonSourceInstall.sh
	\end{lstlisting}
	
	\subsubsection{2 - Manuel Installation}
	\begin{lstlisting}[caption={Installieren von benötigten Packeten und Abhängigkeiten.}]
		elbrus@server:~$ sudo dnf install gcc openssl-devel bzip2-devel\
		libffi-devel zlib-devel wget make -y
	\end{lstlisting}
	
	\begin{lstlisting}[caption={Extrahieren der installierent Dateien.}]
		elbrus@server:~$ tar -xf Python-3.10.2.tar.xz
	\end{lstlisting}
	
	\begin{lstlisting}[caption={Wechseln zu source Verzeichniss. Und ausführen des Konfigurations Scripts.}]
		elbrus@server:~$ cd Python-3.10.0 && ./configure --enable-optimizations
	\end{lstlisting}

	\begin{lstlisting}[caption={Starten des build Prozesses.}]
		elbrus@server:~Python-3.10.0$ cd make -j $(nproc)
	\end{lstlisting}
	
	\begin{lstlisting}[caption={Installieren von Python.}]
		elbrus@server:~Python-3.10.0$ sudo make install
	\end{lstlisting}
	
	\newpage
	\subsection{Rust}
	
	\begin{lstlisting}[caption={Installieren von GNU Compiler Collection.}]
		elbrus@server:~$ sudo dnf install gcc -y
	\end{lstlisting}
	
	\lstset{style=commands}
	\begin{lstlisting}[caption={Installieren von Rust}]
		elbrus@server:~$ curl --proto '=https' --tlsv1.2 -sSf\
		https://sh.rustup.rs/ | sh
		 
		...
		 
		default host triple: x86_64-unknown-linux-gnu
		default toolchain: stable (default)
		profile: default
		modify PATH variable: yes
		
		1) Proceed with installation (default)
		2) Customize installation
		3) Cancel installation
		>1
		
		...
		
		<@\textcolor{codegreen}{stable-x86\_64-unknown-linux-gnu installed}@> - rustc 1.62.1 (e092d0b6b 2022-07-16)
		
		
		Rust is installed now. Great!
		
		To get started you may need to restart your current shell.
		This would reload your PATH environment variable to include
		Cargo's bin directory ($HOME/.cargo/bin).
		
		To configure your current shell, run:
		source "$HOME/.cargo/env"
		elbrus@server:~$
	\end{lstlisting}

	\begin{lstlisting}[caption={Laden der Variablen aus dem Terminal Profil.}]
		elbrus@server:~$ source ~/.profile
	\end{lstlisting}

	\begin{lstlisting}[caption={Hinzufügen des Befehls Cargo zu dem Pfad.}]
		elbrus@server:~$ source ~/.cargo/env
	\end{lstlisting}
	\newpage

	% --------- DATENBANK ---------
	\section{Datenbank}
	\subsection[dependencies]{Voraussetzungen}
	\lstset{style=commands}
	\begin{lstlisting}[caption={Hinzufügen des PostgreSQL Drittanbieter-Repository, um die neuesten PostgreSQL-Pakete zu erhalten.}]
		elbrus@server:~$ sudo yum install\
		https://download.postgresql.org/pub/repos/yum/reporpms/\
		EL-$(rpm -E %{rhel})-x86_64/pgdg-redhat-repo-latest.noarch.rpm
	\end{lstlisting}

	\lstset{style=commands}
	\begin{lstlisting}[caption={Erstellen und berarbeiten des Timescale repository.},keywords={name, baseurl, repo\_gpgcheck, enabled, sslverify, sslcacert, metadata\_expire}, keywordstyle=\color{deepskyblue}]
		<@\textcolor{cobalt}{elbrus@server}@>:~$ sudo tee /etc/yum.repos.d/\ 
		timescale_timescaledb.repo <@\textcolor{amber}{<<EOL}@>
		<@\textcolor{deepskyblue}{[timescale\_timescaledb]}@>
		name=timescale_timescaledb
		baseurl=https://packagecloud.io/timescale/timescaledb\
		/el/$(rpm -E %{rhel})/\$basearch
		repo_gpgcheck=1
		gpgcheck=0
		enabled=1
		<@\textcolor{deepskyblue}{gpgkey}@>=https://packagecloud.io/timescale/timescaledb/gpgkey
		sslverify=1
		sslcacert=/etc/pki/tls/certs/ca-bundle.crt
		metadata_expire=300
		<@\textcolor{amber}{EOL}@>
	\end{lstlisting}

	\lstset{style=commands}
	\begin{lstlisting}[caption={Updaten der lokalen Package-Liste.}]
		elbrus@server:~$ sudo yum update
	\end{lstlisting}

	\begin{lstlisting}[caption={Installieren von TimescaleDB}]
		elbrus@server:~$ sudo dnf -qy module disable postgresql
		elbrus@server:~$ sudo dnf install postgresql14 postgresql14-server -y
		elbrus@server:~$ sudo dnf install timescaledb-2-postgresql-14 -y
	\end{lstlisting}
	
	\newpage
	\subsection[TimescaleDB konfigurieren]{Umgebung Konfigurieren}
	\begin{lstlisting}[caption={Initialisieren der Datenbank.}]
		elbrus@server:~$ /usr/pgsql-14/bin/postgresql-14-setup initdb
	\end{lstlisting}
	
	\begin{lstlisting}[caption={Verknüpfen von 'postgresql' Serive Start mit Serverstart sowie den Service starten.}]
		elbrus@server:~$ sudo systemctl enable postgresql-14
		elbrus@server:~$ sudo systemctl start postgresql-14
	\end{lstlisting}

	\lstset{style=files}
	\begin{lstlisting}[caption={var/lib/pgsql/14/data/postgresql.conf - Ändern der folgenden Zeilen}, numbers=none]
		<@\textcolor{red}{- \#shared\_preload\_libraries = ''}@>
		<@\textcolor{codegreen}{+ shared\_preload\_libraries = 'timescaledb'}@>
		
		<@\textcolor{red}{- \#listen\_addresses = 'localhost'}@>
		<@\textcolor{codegreen}{+ listen\_addresses = '*'}@>
	\end{lstlisting}
	
	\begin{lstlisting}[caption={var/lib/pgsql/14/data/postgresql.conf - Ändern der folgenden Zeilen}, numbers=none]
		# TYPE      DATABASE      USER      ADDRESS      METHOD
		<@\textcolor{codegreen}{+ host}@>       <@\textcolor{codegreen}{elbrus}@>         <@\textcolor{codegreen}{elbrus}@>     <@\textcolor{codegreen}{0.0.0.0/0}@>      <@\textcolor{codegreen}{trust}@>
	\end{lstlisting}
	
	\lstset{style=commands}
	\begin{lstlisting}[caption={Anpassen der Datenbank Einstellungen auf die Server Hardware.}]
		elbrus@server:~$ sudo timescaledb-tune --pg-config=/usr/\
		pgsql-14/bin/pg_config --yes
	\end{lstlisting}

	\begin{lstlisting}[caption={Neustarten des Services um Änderungen zu übernehmen.}]
		elbrus@server:~$ sudo systemctl restart postgresql-14
	\end{lstlisting}

	\newpage
	\subsection{Erstellen der Elbrus-Datenbank}
	\begin{lstlisting}[caption={Verbinden mit dem interaktiven Terminal von 'postgres'.}]
		elbrus@server:~$ sudo su postgres -c psql
	\end{lstlisting}
	Im folgenden Text sind markierte Abschnitte Variablen, welche im darunterliegen SQL gerändert werden können, was aus Sichertsgründen dringend empfohlen wird.
	\begin{enumerate}
		\item Die Datenbank elbrus anlegen
		\item Die Zeitzone auf Europe/Vienna setzen
		\item Den User elbrus mit dem Passwort elbrus123! anlegen
		\item Dem User alle rechte auf die voher erstellte Datenbank geben
	\end{enumerate}	

	\lstset{style=files}
	\begin{lstlisting}[caption={Auführen von SQL Befehlen.}, numbers=none]
		<@\textcolor{amber}{CREATE DATABASE}@> <@\textcolor{darkorchid}{ elbrus}@>;
		<@\textcolor{amber}{ALTER DATABASE}@> elbrus <@\textcolor{amber}{SET}@> timezone TO <@\textcolor{deepskyblue}{'Europe/Vienna'}@>;
		<@\textcolor{amber}{CREATE USER}@> <@\textcolor{darkorchid}{ elbrus}@> PASSWORD <@\textcolor{deepskyblue}{'elbrus123!'}@>;
		<@\textcolor{amber}{GRANT}@> ALL <@\textcolor{amber}{ON}@> DATABASE elbrus TO elbrus;
	\end{lstlisting}
	
	\begin{lstlisting}[caption={Wechseln zu erstellter Datenbank}, numbers=none]
		\c elbrus
	\end{lstlisting}

	\begin{lstlisting}[caption={Hinzufügen der TimescaleDB Erweiterung.}, numbers=none]
		<@\textcolor{amber}{CREATE EXTENSION IF NOT EXISTS}@> timescaledb;
		exit
	\end{lstlisting}

	\newpage
	\subsection{Installation}
	\lstset{style=commands}
	\begin{lstlisting}[caption={Clonen der Software von GitHub.}]
		elbrus@server:~$ cd /var/elbrus
		elbrus@server:~/var/elbrus$ git clone https://github.com/\
		Elbrus-Analytics/database.git
	\end{lstlisting}

	\begin{lstlisting}[caption={Anlegen der benötigten Tabellen duch das ausführen von 'init.sql'.}]
		elbrus@server:~$ psql -U elbrus -d elbrus -f database/sql/init.sql
	\end{lstlisting}
	
	\newpage
	
	% --------- CAPTURE ---------
	\section{Aufzeichnen der Daten}
	\subsection[dependencies]{Voraussetzungen}
	\lstset{style=commands}
	\begin{lstlisting}[caption={Installieren von 'tcpdump' für das aufzeichnen von Daten.}]
		elbrus@server:~$ sudo dnf install tcpdump
	\end{lstlisting}

	\begin{lstlisting}[caption={Anlegen eines Users der Berechtigungen zum ausführen von 'tcpdump' erhält.}]
		elbrus@server:~$ sudo useradd aragog
	\end{lstlisting}
	
	\begin{lstlisting}[caption={Zuweisen von 'tcpdump' zu der Gruppe 'aragog'.}]
		elbrus@server:~$ sudo chgrp aragog /usr/sbin/tcpdump
	\end{lstlisting}
	
	\begin{lstlisting}[caption={Ändern der Berechtigungen auf 'tcpdump'.}]
		elbrus@server:~$ chmod 750 /usr/sbin/tcpdump
		elbrus@server:~$ sudo setcap cap_net_raw,cap_net_admin=eip\
		/usr/sbin/tcpdump
	\end{lstlisting}
	
	\subsection{Installation}
	\begin{lstlisting}[caption={Clonen der Software von GitHub.}]
		elbrus@server:~$ cd /var/elbrus
		elbrus@server:~/var/elbrus$ git clone https://github.com/\
		Elbrus-Analytics/capture-device.git
		elbrus@server:~$ cp capture-device/src/* capture
		elbrus@server:~$ rm -rfd capture-device
	\end{lstlisting}
	
	\newpage
	\subsection[file config]{Umgebung Konfigurieren}
	\subsubsection{1 - Mit Setup Script}
	\begin{lstlisting}[caption={Ausführen des setup Scripts}, breaklines=true,]
		elbrus@server:~$ cd /var/elbrus/capture
		elbrus@server:~/var/elbrus/capture$ bash init.sh
		Do you want to proceed with setup of the 'capture'? (y/n) y
		
		Where should the log be stored (dir) [/var/elbrus/shared/log]:
		Where is the elb-capture-postrotate.sh stored [/var/elbrus/capture/elb-capture-postrotate.sh]:
		Where is the shared config stored [/var/elbrus/shared/.config]:
		
		Should the log be stored at '/var/elbrus/shared/log' ?
		Is the 'elb-capture-postrotate.sh' stored at '/var/elbrus/capture/elb-capture-postrotate.sh' ?
		Is the shared config stored at '/var/elbrus/shared/.config' ? (y/n/exit) y
		#global
		SHAREDCONFIG=/var/elbrus/shared/.config
		
		#paths
		POSTROTATESCRIPT=/var/elbrus/capture/elb-capture-postrotate.sh
		LOGFILE=/var/elbrus/shared/log
		
		#settings
		TIMEPERCAPTURE=900
		MAXFILES=10
		INTERFACE=eth0
		Cleaning up...
		elbrus@server:~/var/elbrus/capture$
	\end{lstlisting}
	
	\subsubsection{2 - Ohne Setup Script}
	\lstset{style=files}
	\begin{lstlisting}[caption={Anhand von '.env.example' eigene '.env' Datei anlegen}, language=bash]
		#global
		SHAREDCONFIG=/var/elbrus/shared/.config
		
		#settings
		TIMEPERCAPTURE=900
		MAXFILES=10
		INTERFACE=eth0
		
		#path
		POSTROTATESCRIPT=/var/elbrus/capture/elb-capture-postrotate.sh
		LOGFILE=/var/elbrus/capture/capture"-$(date +"%Y-%U")".log
	\end{lstlisting}

	\newpage
	\subsection[systemd service]{Der Systemd Service}
	\begin{lstlisting}[caption={capture.service.example - Die Variable 'WorkingDirectory', Die Variable 'User' sowie die Variable 'ExecStopPost' anpassen.},language=bash ,keywords={WorkingDirectory, User, ExecStopPost}, keywordstyle=\color{red}, firstnumber=3]
		...
		#job is starting immediatly after the start action has been called
		Type=simple
		#the user to execute the script
		User=aragog
		#the working directory
		WorkingDirectory=/var/elbrus/capture
		#which script should be executed
		ExecStart=/bin/bash elb-capture.sh
		#when the script should restart
		Restart=on-failure
		#set the restart timeout
		RestartSec=5
		#which script should be executed when the service stops
		ExecStopPost=/bin/bash elb-capture-log.sh
		
		[Install]
		...
	\end{lstlisting}
	
	\lstset{style=commands}
	\begin{lstlisting}[caption={Kopieren des Serviceprogrammes}]
		elbrus@server:~$ cp capture.service.example\
		/etc/systemd/system/capture.service
	\end{lstlisting}

	\begin{lstlisting}[caption={Neuladen des 'systemctl' Deamons}]
		elbrus@server:~$ systemctl daemon-reload
	\end{lstlisting}
	
	\begin{lstlisting}[caption={Aktivieren des Serviceprogrammes}]
		elbrus@server:~$ systemctl enable capture.service
	\end{lstlisting}

	\begin{lstlisting}[caption={Starten des Serviceprogrammes}]
		elbrus@server:~$ systemctl start capture.service
	\end{lstlisting}
	
	\newpage
	
	
	% --------- PACKET CAPTURE IMPORTER ---------
	\section{Packet Capture Importer}
	\subsection{Installation}
	\begin{lstlisting}[caption={Clonen der Software von GitHub.}]
		elbrus@server:~$ cd /var/elbrus
		elbrus@server:~/var/elbrus$ git clone https://github.com/\
		Elbrus-Analytics/database.git
	\end{lstlisting}

	\subsection[file config]{Umgebung Konfigurieren}
	\subsubsection{1 - Mit Setup Script}
	\begin{lstlisting}[caption={Ausführen des setup Scripts}, breaklines=true,]
		elbrus@server:~$ cd /var/elbrus
		elbrus@server:~/var/elbrus$ bash database/importer/pcap-importer/install.sh
		Do you want to proceed? (y/n) y
		
		Where is the shared config stored [/var/elbrus/shared/.config]: /var/elbrus/shared/.config
		Where is the 'pcap-importer' (dir) stored [/var/elbrus/pcap-importer]: /var/elbrus/pcap-importer
		
		Would you like to store the 'pcap-importer' at '/var/elbrus/pcap-importer' ?
		Is the shared config stored at '/var/elbrus/shared/.config' ? (y/n/exit) y
		Submodule 'importer/pcap-importer/pcap-analyzer' (https://github.com/rusticata/pcap-analyzer.git) registered for path 'importer/pcap-importer/pcap-analyzer'
		Cloning into '/var/elbrus/database/importer/pcap-importer/pcap-analyzer'...
		Submodule path 'importer/pcap-importer/pcap-analyzer': checked out '26abc0b0f4d9b2f0e6a72a62e694cd60ae6b6011'
		Start Building ... (this may take a while)
		<@\textcolor{codegreen}{Compiling}@> proc-macro2 v1.0.38
		<@\textcolor{codegreen}{Compiling}@> unicode-xid v0.2.3
		<@\textcolor{codegreen}{Compiling}@> syn v1.0.93
		...
		<@\textcolor{codegreen}{Compiling}@> libpcap-tools v0.1.0 (/var/elbrus/database/importer/pcap-importer/pcap-analyzer/libpcap-tools)
		<@\textcolor{codegreen}{Compiling}@> tokio-postgres v0.7.6
		<@\textcolor{codegreen}{Compiling}@> pcap-importer v0.1.0 (/var/elbrus/database/importer/pcap-importer)
		Finished release [optimized] target(s) in 1m 38s
		Cleaning up...
		elbrus@server:~/var/elbrus$
	\end{lstlisting}
	
	\newpage
	\subsubsection{2 - Ohne Setup Script}	
	
	\lstset{style=files}
	\begin{lstlisting}[caption={pcap-importer/.env - Anpassen an eigene Werte.}, language=bash]
		#where the traces should be stored
		PCAPFOLDER=/var/elbrus/shared/traces/
		
		#where the importer should be stored
		IMPORTERPATH=/var/elbrus/pcap-importer
		
		#database values
		DB_HOST=10.0.76.220
		DB_PORT=5432
		DB_NAME=elbrus
		DB_USER=elbrus
		DB_PASSWORD=elbrus123!
	\end{lstlisting}
	\newpage
	
	% --------- REPORT GENERATOR ---------
	\section{Report Generator}
	\lstset{style=commands}
	\subsection{Installation}
	\begin{lstlisting}[caption={Clonen der Software von GitHub.}]
		elbrus@server:~$ cd /var/elbrus
		elbrus@server:~/var/elbrus$ git https://github.com/\
		Elbrus-Analytics/report-generator.git
	\end{lstlisting}

	\subsection[file config]{Umgebung Konfigurieren}
	
	\subsubsection{1 - Mit Setup Script}
	
	\begin{lstlisting}[caption={Ausführen des 'install.sh' Scripts.}]
		elbrus@server:~$ bash report-generator/install.sh
		Do you want to proceed with setup of the 'report-generator'? (y/n) y
		
		Where is the shared config stored [/var/elbrus/shared/.config]:
		
		Is the shared config stored at '/var/elbrus/shared/.config' ? (y/n/exit) y
		Install dependencies ...
		
		...
		
		elbrus@server:~$
	\end{lstlisting}
	
	\newpage
	
	% --------- SNMP MANAGER ---------
	\section{SNMP Manager}
	\newpage
	
	% --------- SSH MANAGER ---------
	\section{SSH Manager}
	\subsection[file config]{Umgebung Konfigurieren}
	
	Kopieren von 'requirements.txt', '.env.example', 'initialise.sh', 'routine.sh', 'setup.sh', 'main.py', 'ssh-manager.service.example', 'ssh-manager-schedule.timer.example' in den selben beliebigen Ordner. 
	
	\lstset{style=files}
	\begin{lstlisting}[caption={Anhand von '.env.example' eigene '.env' Datei anlegen}, language=bash]
		#values regarding the jumpserver:
		#IP, PORT and USER values must be set!
		#depending on the usage you can set either:
		#   -PASS and KEYFILE: keyfile is used with passphrase
		#   -only KEYFILE: the keyfile is used
		#   -only PASS: the password is used as is regular credentials
		JUMPSERVER_IP="2.2.2.15"
		JUMPSERVER_PORT=22
		JUMPSERVER_USER=admin
		JUMPSERVER_PASS=password
		SSH_KEYFILE='my/sample/path'
		
		#all database values must be set!
		POSTGRES_HOST="192.168.0.1"
		POSTGRES_PORT=245
		POSTGRES_DB=mydb
		POSTGRES_USER=admin
		POSTGRES_PASS=password
		
		#paths are configured by running 'setup.sh'
		CONFIGPATH="/thats/where/i/store/my/configs"
		MAINPATH="/the/path/to/main.py"
	\end{lstlisting}
	
	\newpage
	\subsubsection{1 - Mit Setup Script}
	\lstset{style=commands}
	\begin{lstlisting}[caption={Ausführen des setup Scripts}]
		elbrus@server:~$ cd ssh-manager/src
		elbrus@server:~/ssh-manager/src$ ./setup.sh
		Setup for ssh-manager
		Do you want to proceed? (y/n) y
		we will proceed
		
		Where do you want the config to be stored: (abolut path) /my/sample/path
		Where is the 'main.py' file stored: (abolut path) /path/to/main.py
		
		Do you want to store the config files at "/my/sample/path"? 
		Is your 'main.py' stored at "/path/to/main" (y/n/exit) y
		
		The paths have been set!
		
		Do you want to configure the systemd Service? (y/n/exit) y
		
		Which User should execute the Service? elbrus
		
		The systemd Service has been configured!
		
		Do you want to run the initialise script? (y/n/exit) y
		
		...
		
		finished setup
		
		elbrus@server:~/ssh-manager/src$
	\end{lstlisting}
	
	\newpage
	\subsubsection{2 - Ohne Setup Script}
	\lstset{style=files}
	\begin{lstlisting}[caption={.env - Die Variable 'CONFIGPATH' sowie die Variable 'MAINPATH' anpassen.}, language=bash, keywords={CONFIGPATH, MAINPATH}, keywordstyle=\color{red}, firstnumber=17]
		...
		POSTGRES_PASS=password
		
		#paths are configured by running 'setup.sh'
		CONFIGPATH="/thats/where/i/store/my/configs"
		MAINPATH="/the/path/to/main.py"
	\end{lstlisting}

	\begin{lstlisting}[caption={ssh-manager.service.example - Die Variable 'WorkingDirectory' sowie die Variable 'User' anpassen.},language=bash ,keywords={WorkingDirectory, User}, keywordstyle=\color{red}, firstnumber=5]
		...
		#job is starting immediatly after the start action has been called
		Type=simple
		#the user to execute the script
		User=elbrus
		#the working directory
		WorkingDirectory=/home/elbrus/Desktop/ssh-manager/src/
		#which script should be executed
		ExecStart=/bin/bash routine.sh
		...
	\end{lstlisting}

	\lstset{style=commands}	
	\begin{lstlisting}[caption={Ausführen des Scripts zur Initialisierung des VCS Verzeichnisses.}]
		elbrus@server:~$ ssh-manager/src/initialise.sh
	\end{lstlisting}

	\subsection[dependencies]{Voraussetzungen}
	\begin{lstlisting}[caption={Installieren von fehlenden python3 Packages.}]
		elbrus@server:~$ pip3 install -r ssh-manager/requirements.txt
	\end{lstlisting}

	\newpage
	\subsection[systemd service]{Der Systemd Service}
	\begin{lstlisting}[caption={Kopieren des Serviceprogrammes}]
		elbrus@server:~$ cp src/ssh-manager.service.example\
		/etc/systemd/system/ssh-manager.service
	\end{lstlisting}

	\begin{lstlisting}[caption={Kopieren des Zeitplanungsprogrammes.}]
		elbrus@server:~$ cp src/ssh-manager-schedule.timer.example\
		/etc/systemd/system/ssh-manager-schedule.timer
	\end{lstlisting}

	\begin{lstlisting}[caption={Neuladen des 'systemctl' Deamons}]
		elbrus@server:~$ systemctl daemon-reload
	\end{lstlisting}

	\begin{lstlisting}[caption={Aktivieren des Serviceprogrammes}]
		elbrus@server:~$ systemctl enable ssh-manager.service
	\end{lstlisting}

	\begin{lstlisting}[caption={Aktivieren des Zeitplanungsprogrammes}]
		elbrus@server:~$ systemctl enable ssh-manager-schedule.timer
	\end{lstlisting}

	\begin{lstlisting}[caption={Aktivieren des Zeitplanungsprogrammes}]
		elbrus@server:~$ systemctl enable ssh-manager-schedule.timer
	\end{lstlisting}

	\begin{lstlisting}[caption={Starten des Zeitplanungsprogrammes}]
		elbrus@server:~$ systemctl start ssh-manager-schedule.timer
	\end{lstlisting}
	\newpage
	
	% --------- API ---------
	\section{API}
	\newpage
	
	\section{Webinterface}
	\newpage
\end{document}