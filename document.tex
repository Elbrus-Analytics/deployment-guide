\documentclass{article}
\usepackage[utf8]{inputenc}

%%FOR CODES
\usepackage{listings,lstautogobble}
\usepackage{xcolor}
\usepackage{caption}

%%COLORES
\definecolor{realylightgray}{gray}{0.95}
\definecolor{RoyalBlue}{cmyk}{1, 0.50, 0, 0}
\definecolor{capri}{rgb}{0.0, 0.75, 1.0}
\definecolor{deepskyblue}{rgb}{0.0, 0.75, 1.0}
\definecolor{cobalt}{rgb}{0.0, 0.28, 0.67}
\definecolor{codegreen}{rgb}{0,0.6,0}
\definecolor{amber}{rgb}{1.0, 0.49, 0.0}
\definecolor{darkorchid}{rgb}{0.6, 0.2, 0.8}

\lstset{
	showstringspaces=false,
	showtabs=false,
	showspaces=false,
	stringstyle=\ttfamily,
	frameround=ffff,
	frame=single,
	framextopmargin=4pt,
	framexbottommargin=4pt,
	rulecolor=\color{darkgray},
	gobble=12, %align text inside box left <-> right
	belowskip=2em,
	keepspaces=true,
	escapeinside={<@}{@>},
	columns=fullflexible
}

%%template for commands
\lstdefinestyle{commands}{
	basicstyle=\scriptsize\ttfamily,
	commentstyle=\ttfamily\itshape\color{gray},
	backgroundcolor=\color{realylightgray},
	keywords=[1]{elbrus@server, root@server},
	keywordstyle=[1]\color{cobalt},
	numbers=none,
	breaklines=false
}

%%template for files
\lstdefinestyle{files}{
	basicstyle=\ttfamily\footnotesize,
	commentstyle=\color{codegreen},
	backgroundcolor=\color{realylightgray},    
	deletekeywords=[1]{elbrus@server, root@server},                                       
	numbers=left,       
	numbersep=5pt,
	breaklines=true,                                                 
}

\captionsetup[lstlisting]{singlelinecheck=false, } %to align caption left
\lstset{style=commands}

\begin{document}
	\title{Elbrus Analytics - Bereitstellungshandbuch}
	\author{Tobias Schmidt}
	\date{\today}
	\maketitle
	\newpage
	
	Bekanntes Problem: Beim herauskopieren von Befehlen wird das Apostrophe Zeichen falsch kopiert und führt zu Eingabe Störungen. Lösung: Apostrophe Zeichen des kopierten Befehls händisch im Terminal mit Apostrophe Zeichen austauschen. 
	
	\section[Server Infrastruktur]{Server Infrastruktur}
	\subsection{SSH-Zugriff vorbereiten}
	
	\lstset{style=commands}
	\begin{lstlisting}[caption={Updaten vorhandener Packages.}]
		root@server:~$ yum update -y
	\end{lstlisting}
	
	\lstset{style=commands}
	\begin{lstlisting}[caption={Installieren des 'ssh' Packages.}]
		root@server:~$ yum install -y openssh-server
	\end{lstlisting}

	\lstset{style=commands}
	\begin{lstlisting}[caption={Starten des 'sshd' Services.}]
		root@server:~$ systemctl start sshd
	\end{lstlisting}

	\lstset{style=commands}
	\begin{lstlisting}[caption={Aktivieren des 'sshd' Services.}]
		root@server:~$ systemctl enable sshd
	\end{lstlisting}

	\lstset{style=commands}
	\begin{lstlisting}[caption={Anlegen des Users Elbrus.}]
		root@server:~$ useradd elbrus
	\end{lstlisting}

	\lstset{style=commands}
	\begin{lstlisting}[caption={Hinzufügen des Users Elbrus zu der Gruppe 'wheel'.}]
		root@server:~$ usermod -aG wheel elbrus
	\end{lstlisting}

	\lstset{style=commands}
	\begin{lstlisting}[caption={Ändern des Passwords für den User Elbrus.}]
		root@server:~$ passwd elbrus
		Changing password for user elbrus.
		New password:
		Retype new password:
		passwd: all authentication tokens updated successfully
		root@server:~$
	\end{lstlisting}

	\lstset{style=commands}
	\begin{lstlisting}[caption={Wechseln zu User elbrus.}]
		root@server:~$ su elbrus
	\end{lstlisting}
	\newpage
	
	\subsection{Initiale Server Konfiguration}

	\lstset{style=commands}
	\begin{lstlisting}[caption={Setzen der Zeitzone auf 'Europa/Wien'.}]
		elbrus@server:~$ sudo timedatectl set-timezone Europe/Vienna
	\end{lstlisting}

	\lstset{style=commands}
	\begin{lstlisting}[caption={Installieren von dem 'firewalld' Service.}]
		elbrus@server:~$ sudo dnf install firewalld
	\end{lstlisting}
	
	\subsubsection{Node.Js}
	
	\lstset{style=commands}
	\begin{lstlisting}[caption={Installieren des Framworks 'Node.Js'.}]
		elbrus@server:~$ sudo dnf -y module install nodejs:12
	\end{lstlisting}
	
	\subsubsection{Ablagestruktur}
	
	\lstset{style=commands}
	\begin{lstlisting}[caption={Anlegen der Verzeichnissstruktur.}]
		elbrus@server:~$ sudo mkdir /var/elbrus/shared/keys
		elbrus@server:~$ sudo chown -R elbrus:elbrus /var/elbrus
		elbrus@server:~$ cd /var/elbrus
		elbrus@server:/var/elbrus$ chmod -R 777 /var/elbrus/shared
	\end{lstlisting}
	\newpage

	% -------- Git Repos clonen
	\subsection{Git}
	
	\lstset{style=commands}
	\begin{lstlisting}[caption={Installieren von dem VCS 'git'.}]
		elbrus@server:~$ sudo yum install -y git
	\end{lstlisting}
	
	\lstset{style=commands}
	\subsubsection{Git - Erstellen der SSH-Keys}
	\begin{lstlisting}[caption={Wechseln des Verzeichnisses.}]
		elbrus@server:~$ cd /var/elbrus/shared/keys
	\end{lstlisting}
	
	\lstset{style=commands}
	\begin{lstlisting}[caption={Erstellen des SSH-keys der für das Herunterladen der 'Database' benötigt wird.}]
		elbrus@server:/var/elbrus/keys$ ssh-keygen -t rsa -b 2048 -f database_key -q -N ""
	\end{lstlisting}

	\lstset{style=commands}
	\begin{lstlisting}[caption={Erstellen des SSH-keys der für das Herunterladen des 'Capture-Device' benötigt wird.}]
		elbrus@server:/var/elbrus/keys$ ssh-keygen -t rsa -b 2048 -f capture_device_key -q -N ""
	\end{lstlisting}

	\lstset{style=commands}
	\begin{lstlisting}[caption={Erstellen des SSH-keys der für das Herunterladen des 'Report-Generator' benötigt wird.}]
		elbrus@server:/var/elbrus/keys$ ssh-keygen -t rsa -b 2048 -f \
		 report_generator_key -q -N ""
	\end{lstlisting}

	\lstset{style=commands}
	\begin{lstlisting}[caption={Erstellen des SSH-keys der für das Herunterladen des 'SNMP-Managers' benötigt wird.}]
		elbrus@server:/var/elbrus/keys$ ssh-keygen -t rsa -b 2048 -f snmp_manager_key -q -N ""
	\end{lstlisting}

	\lstset{style=commands}
	\begin{lstlisting}[caption={Erstellen des SSH-keys der für das Herunterladen des 'SSH-Managers' benötigt wird.}]
		elbrus@server:/var/elbrus/keys$ ssh-keygen -t rsa -b 2048 -f ssh_manager_key -q -N ""
	\end{lstlisting}
	
	\lstset{style=commands}
	\begin{lstlisting}[caption={Erstellen des SSH-keys der für das Herunterladen des 'Uptime-Monitors' benötigt wird.}]
		elbrus@server:/var/elbrus/keys$ ssh-keygen -t rsa -b 2048 -f uptime_monitor_key -q -N ""
	\end{lstlisting}
	
	\lstset{style=commands}
	\begin{lstlisting}[caption={Erstellen des SSH-keys der für das Herunterladen des 'geo session finders' benötigt wird.}]
		elbrus@server:/var/elbrus/keys$ ssh-keygen -t rsa -b 2048 \
		 -f geo_session_finder_key -q -N ""
	\end{lstlisting}

	\lstset{style=commands}
	\begin{lstlisting}[caption={Erstellen des SSH-keys der für das Herunterladen des 'office365 analyzers' benötigt wird.}]
		elbrus@server:/var/elbrus/keys$ ssh-keygen -t rsa -b 2048 \
		-f office365_analyzer_key -q -N ""
	\end{lstlisting}
	
	\lstset{style=commands}
	\begin{lstlisting}[caption={Erstellen des SSH-keys der für das Herunterladen der 'API' benötigt wird.}]
		elbrus@server:/var/elbrus/keys$ ssh-keygen -t rsa -b 2048 -f api_key -q -N ""
	\end{lstlisting}
	
	\lstset{style=commands}
	\begin{lstlisting}[caption={Erstellen des SSH-keys der für das Herunterladen des 'Webinterfaces' benötigt wird.}]
		elbrus@server:/var/elbrus/keys$ ssh-keygen -t rsa -b 2048 -f webinterface_key -q -N ""
	\end{lstlisting}

	Bevor mit der Installation vorgefahren werden kann müssen die soeben angelegten SSH-Keys an "keys@Elbrus-Analytics.at" gesendet werden. Bitte beachten Sie, dass Ihre Email-Adresse Sie als berechtigten Nutzer ausweist.
	
	\subsubsection{Git - Clonen der Software}

	\lstset{style=commands}
	\begin{lstlisting}[caption={Wechseln des Verzeichnisses.}]
		elbrus@server:~$ cd /var/elbrus
	\end{lstlisting}
	
	\lstset{style=commands}
	\begin{lstlisting}[caption={Clonen der Datenbank Software.}]
		elbrus@server:/var/elbrus$ git clone git@github.com:Elbrus-Analytics/database.git\
		 --config core.sshCommand="ssh -i /var/elbrus/shared/keys/database_key"
	\end{lstlisting}
	
	\lstset{style=commands}
	\begin{lstlisting}[caption={Clonen der 'Capture-Device' Software.}]
		elbrus@server:/var/elbrus$ git clone git@github.com:Elbrus-Analytics/capture-device.git\
		 --config core.sshCommand="ssh -i /var/elbrus/shared/keys/capture_device_key"
	\end{lstlisting}
	
	%\lstset{style=commands}
	%\begin{lstlisting}[caption={Clonen der Packet-Importer Software}]
	%	elbrus@server:/var/elbrus$ git clone https://github.com/\
	%	Elbrus-Analytics/database.git
	%\end{lstlisting}

	\lstset{style=commands}
	\begin{lstlisting}[caption={Clonen der 'Report-Generator' Software.}]
		elbrus@server:/var/elbrus$ git clone git@github.com:Elbrus-Analytics/\
		report-generator.git --config core.sshCommand="ssh -i /var/elbrus/shared/keys/report_generator_key"
	\end{lstlisting}
	
	\lstset{style=commands}
	\begin{lstlisting}[caption={Clonen der 'SNMP-Manager' Software.}]
		elbrus@server:/var/elbrus$ git clone git@github.com:Elbrus-Analytics/snmp-manager.git\
		 --config core.sshCommand="ssh -i /var/elbrus/shared/keys/snmp_manager_key"
	\end{lstlisting}

	\lstset{style=commands}
	\begin{lstlisting}[caption={Clonen der 'SSH-Manager' Software.}]
		elbrus@server:/var/elbrus$ git clone git@github.com:Elbrus-Analytics/ssh-manager.git\
		 --config core.sshCommand="ssh -i /var/elbrus/shared/keys/ssh_manager_key"
	\end{lstlisting}
	
	\lstset{style=commands}
	\begin{lstlisting}[caption={Clonen der 'Uptime-Monitor' Software.}]
		elbrus@server:/var/elbrus$ git clone git@github.com:Elbrus-Analytics/uptime_monitor.git\
		 --config core.sshCommand="ssh -i /var/elbrus/shared/keys/uptime_monitor_key"
	\end{lstlisting}

	\lstset{style=commands}
	\begin{lstlisting}[caption={Clonen der 'geo session finders' Software.}]
		elbrus@server:/var/elbrus/keys$ git clone git@github.com:Elbrus-Analytics/\
		geo_session_finder.git --config core.sshCommand="ssh -i /var/elbrus/shared/keys/geo_session_finder_key"
	\end{lstlisting}

	\lstset{style=commands}
	\begin{lstlisting}[caption={Clonen der 'office365-analyzer' Software.}]
		elbrus@server:/var/elbrus/keys$ git clone git@github.com:Elbrus-Analytics/\
		office365-analyzer.git --config core.sshCommand="ssh -i /var/elbrus/shared/keys/office365_analyzer_key"
	\end{lstlisting}
	
	\lstset{style=commands}
	\begin{lstlisting}[caption={Clonen der API Software.}]
		elbrus@server:/var/elbrus$ git clone git@github.com:Elbrus-Analytics/api.git\
		 --config core.sshCommand="ssh -i /var/elbrus/shared/keys/api_key"
	\end{lstlisting}

	\lstset{style=commands}
	\begin{lstlisting}[caption={Clonen der Packet-Importer Software}]
		elbrus@server:/var/elbrus$ git clone git@github.com:Elbrus-Analytics/webinterface.git\
		 --config core.sshCommand="ssh -i /var/elbrus/shared/keys/webinterface_key"
	\end{lstlisting}
	\newpage
	
	\subsection{Python}
	\subsubsection{1 - Automatische Installation}

	\lstset{style=commands}
	\begin{lstlisting}[caption={Ausführen des 'pythonSourceInstall.sh' Scripts.}]
		elbrus@server:/var/elbrus$ sudo bash report-generator/pythonSourceInstall.sh
	\end{lstlisting}
	
	\subsubsection{2 - Manuel Installation}
	
	\lstset{style=commands}
	\begin{lstlisting}[caption={Installieren von benötigten Packeten und Abhängigkeiten.}]
		elbrus@server:~$ cd /var/elbrus
		elbrus@server:/var/elbrus$ sudo dnf install gcc openssl-devel bzip2-devel\
		libffi-devel zlib-devel wget make -y
	\end{lstlisting}
	
	\lstset{style=commands}
	\begin{lstlisting}[caption={Herunterladen der Source Datei.}]
		elbrus@server:/var/elbrus$ wget https://www.python.org/ftp/python/\
		3.10.0/Python-3.10.0.tar.xz
	\end{lstlisting}
	
	\lstset{style=commands}
	\begin{lstlisting}[caption={Extrahieren der installierten Datei.}]
		elbrus@server:/var/elbrus$ tar -xf Python-3.10.2.tar.xz
	\end{lstlisting}
	
	\lstset{style=commands}
	\begin{lstlisting}[caption={Wechseln zu source Verzeichniss. Und ausführen des Konfigurations Scripts.}]
		elbrus@server:/var/elbrus$ cd Python-3.10.0 && ./configure --enable-optimizations
	\end{lstlisting}

	\lstset{style=commands}
	\begin{lstlisting}[caption={Starten des build Prozesses.}]
		elbrus@server:/var/elbrus/Python-3.10.0$ cd make -j $(nproc)
	\end{lstlisting}
	
	\lstset{style=commands}
	\begin{lstlisting}[caption={Installieren von Python.}]
		elbrus@server:/var/elbrus/Python-3.10.0$ sudo make install
	\end{lstlisting}

	\lstset{style=commands}
	\begin{lstlisting}[caption={Löschen der kompremierten Python Datei.}]
		elbrus@server:/var/elbrus/Python-3.10.0$ cd .. && rm Python-3.10.0.tar.xz
	\end{lstlisting}

	\subsubsection{Upgrade von 'pip'}

	\lstset{style=commands}
	\begin{lstlisting}[caption={Upgraden von 'pip'.}]
		elbrus@server:~$ /usr/local/bin/python3.10 -m pip install --upgrade pip
	\end{lstlisting}

	\subsection{Rust}
	
	\lstset{style=commands}
	\begin{lstlisting}[caption={Installieren von GNU Compiler Collection.}]
		elbrus@server:~$ sudo dnf install gcc -y
	\end{lstlisting}
	
	\lstset{style=commands}
	\begin{lstlisting}[caption={Installieren von Rust.}]
		elbrus@server:~$ curl --proto '=https' --tlsv1.2 -sSf \
		https://sh.rustup.rs/ | sh
		 
		...
		 
		default host triple: x86_64-unknown-linux-gnu
		default toolchain: stable (default)
		profile: default
		modify PATH variable: yes
		
		1) Proceed with installation (default)
		2) Customize installation
		3) Cancel installation
		>1
		
		...
		
		<@\textcolor{codegreen}{stable-x86\_64-unknown-linux-gnu installed}@> - rustc 1.62.1 (e092d0b6b 2022-07-16)
		
		
		Rust is installed now. Great!
		
		To get started you may need to restart your current shell.
		This would reload your PATH environment variable to include
		Cargo's bin directory ($HOME/.cargo/bin).
		
		To configure your current shell, run:
		source "$HOME/.cargo/env"
		elbrus@server:~$
	\end{lstlisting}
	
	\lstset{style=commands}
	\begin{lstlisting}[caption={Laden der Variablen aus dem Terminal Profil.}]
		elbrus@server:~$ source ~/.profile
	\end{lstlisting}

	\lstset{style=commands}
	\begin{lstlisting}[caption={Hinzufügen des Befehls Cargo zu dem Pfad.}]
		elbrus@server:~$ source ~/.cargo/env
	\end{lstlisting}
	\newpage
	
	% --------- Verbindung wenn Software auf verschiedene Server aufgeteilt ist ------------
	%\section{SSH-Keys}
	%\input{SSH-Keys.tex}
	%\newpage
	
	% --------- DATENBANK ---------
	\section{Datenbank}
	% --------- deployment guide for the database ---------
	\subsection[dependencies]{Voraussetzungen}
	
	\lstset{style=commands}
	\begin{lstlisting}[caption={Hinzufügen des PostgreSQL Drittanbieter-Repository, um die neuesten PostgreSQL-Pakete zu erhalten.}]
		elbrus@server:~$ sudo yum install\
		https://download.postgresql.org/pub/repos/yum/reporpms/\
		EL-$(rpm -E %{rhel})-x86_64/pgdg-redhat-repo-latest.noarch.rpm
	\end{lstlisting}
	
	\lstset{style=commands}
	\begin{lstlisting}[caption={Erstellen des Timescale repository.}]
		elbrus@server:~$ sudo tee /etc/yum.repos.d/timescale_timescaledb.repo <<EOL
		[timescale_timescaledb]
		name=timescale_timescaledb
		baseurl=https://packagecloud.io/timescale/timescaledb/el/$(rpm -E %{rhel})/$basearch
		repo_gpgcheck=1
		gpgcheck=0
		enabled=1
		gpgkey=https://packagecloud.io/timescale/timescaledb/gpgkey
		sslverify=1
		sslcacert=/etc/pki/tls/certs/ca-bundle.crt
		metadata_expire=300
		EOL
	\end{lstlisting}
	
	\lstset{style=commands}
	\begin{lstlisting}[caption={Updaten der lokalen Package-Liste.}]
		elbrus@server:~$ sudo yum update -y
	\end{lstlisting}
	
	\lstset{style=commands}
	\begin{lstlisting}[caption={Installieren von TimescaleDB.}]
		elbrus@server:~$ sudo dnf -qy module disable postgresql
		elbrus@server:~$ sudo dnf install postgresql14 postgresql14-server -y
		elbrus@server:~$ sudo dnf install timescaledb-2-postgresql-14 -y
	\end{lstlisting}
	\newpage
	
	\subsection[TimescaleDB konfigurieren]{Umgebung Konfigurieren}
	
	\lstset{style=commands}
	\begin{lstlisting}[caption={Initialisieren der Datenbank.}]
		elbrus@server:~$ sudo /usr/pgsql-14/bin/postgresql-14-setup initdb
	\end{lstlisting}
	
	\lstset{style=commands}
	\begin{lstlisting}[caption={Verknüpfen von 'postgresql' Serive Start mit Serverstart sowie den Service starten.}]
		elbrus@server:~$ sudo systemctl enable postgresql-14
		elbrus@server:~$ sudo systemctl start postgresql-14
	\end{lstlisting}
	
	\lstset{style=files}
	\begin{lstlisting}[caption={/var/lib/pgsql/14/data/postgresql.conf - Ändern der folgenden Zeilen.}, numbers=none]
		<@\textcolor{red}{- \#shared\_preload\_libraries = ''}@>
		<@\textcolor{codegreen}{+ shared\_preload\_libraries = 'timescaledb'}@>
		
		<@\textcolor{red}{- \#listen\_addresses = 'localhost'}@>
		<@\textcolor{codegreen}{+ listen\_addresses = '*'}@>
	\end{lstlisting}
	
	\lstset{style=files}
	\begin{lstlisting}[caption={/var/lib/pgsql/14/data/pg\_hba.conf - Ändern der folgenden Zeilen.}, numbers=none]
		# TYPE      DATABASE      USER      ADDRESS      METHOD
		<@\textcolor{codegreen}{+ host}@>      <@\textcolor{codegreen}{elbrus}@>        <@\textcolor{codegreen}{elbrus}@>    <@\textcolor{codegreen}{0.0.0.0/0}@>    <@\textcolor{codegreen}{trust}@>
	\end{lstlisting}
	
	\lstset{style=commands}
	\begin{lstlisting}[caption={Anpassen der Datenbank Einstellungen auf die Server Hardware.}]
		elbrus@server:~$ sudo timescaledb-tune --pg-config=/usr/\
		pgsql-14/bin/pg_config --yes
	\end{lstlisting}
	
	\lstset{style=commands}
	\begin{lstlisting}[caption={Neustarten des Services um Änderungen zu übernehmen.}]
		elbrus@server:~$ sudo systemctl restart postgresql-14
	\end{lstlisting}
	\newpage
	
	\subsection{Erstellen der Elbrus-Datenbank}
	
	\lstset{style=commands}
	\begin{lstlisting}[caption={Verbinden mit dem interaktiven Terminal von 'postgres'.}]
		elbrus@server:~$ sudo su postgres -c psql
	\end{lstlisting}
	Im folgenden Text sind markierte Abschnitte Variablen, welche im darunterliegen SQL gerändert werden können, was aus Sichertsgründen dringend empfohlen wird.
	\begin{enumerate}
		\item Die Datenbank elbrus anlegen
		\item Die Zeitzone auf Europe/Vienna setzen
		\item Den User elbrus mit dem Passwort elbrus123! anlegen
		\item Dem User alle rechte auf die voher erstellte Datenbank geben
	\end{enumerate}	
	
	\lstset{style=files}
	\begin{lstlisting}[caption={Auführen von SQL Befehlen.}, numbers=none]
		<@\textcolor{amber}{CREATE DATABASE}@> <@\textcolor{darkorchid}{ elbrus}@>;
		<@\textcolor{amber}{ALTER DATABASE}@> elbrus <@\textcolor{amber}{SET}@> timezone TO <@\textcolor{deepskyblue}{'Europe/Vienna'}@>;
		<@\textcolor{amber}{CREATE USER}@> <@\textcolor{darkorchid}{ elbrus}@> PASSWORD <@\textcolor{deepskyblue}{'elbrus123!'}@>;
		<@\textcolor{amber}{GRANT}@> ALL <@\textcolor{amber}{ON}@> DATABASE elbrus TO elbrus;
	\end{lstlisting}
	
	\begin{lstlisting}[caption={Wechseln zu erstellter Datenbank.}, numbers=none]
		\c elbrus
	\end{lstlisting}
	
	\begin{lstlisting}[caption={Hinzufügen der TimescaleDB Erweiterung.}, numbers=none]
		<@\textcolor{amber}{CREATE EXTENSION IF NOT EXISTS}@> timescaledb;
		exit
	\end{lstlisting}
	
	\lstset{style=commands}
	\begin{lstlisting}[caption={Anlegen der benötigten Tabellen duch das ausführen von 'init.sql'.}]
		elbrus@server:/var/elbrus$ psql -U elbrus -d elbrus -f \
		database/sql/init.sql
	\end{lstlisting}
	\newpage
	
	% ----------- Globale Konfiguration ---------
	\section{Globale Konfiguration}
	% --------- deployment guide for the global konfiguration ---------
	\lstset{style=files}
	\begin{lstlisting}[caption={Anhand von '.config.example' eigene '.config' Datei in \newline'/var/elbrus/shared' anlegen.}, language=bash]
		#database settings
		DB_HOST=localhost
		DB_PORT=5432
		DB_NAME=elbrus
		DB_USER=elbrus
		DB_PASSWORD=elbrus123!
		
		#paths
		PCAPFOLDER=/var/elbrus/shared/traces
		IMPORTERPATH=/var/elbrus/pcap-importer/pcap-importer
		REPORTERPATH=/var/elbrus/report-generator/src/main.py
	\end{lstlisting}
	
	\lstset{style=commands}
	\begin{lstlisting}
		elbrus@server:/var/elbrus$ sudo chown elbrus:elbrus /var/elbrus/shared/.config
		elbrus@server:/var/elbrus$ sudo chmod 776 /var/elbrus/shared/.config
	\end{lstlisting}
	\newpage
	
	% --------- CAPTURE ---------
	\section{Aufzeichnen der Daten}
	% --------- deployment guide for the capture ---------
	\subsection{Voraussetzungen}
	
	\lstset{style=commands}
	\begin{lstlisting}[caption={Installieren von 'tcpdump' für das aufzeichnen von Daten.}]
		elbrus@server:~$ sudo dnf install tcpdump
	\end{lstlisting}
	
	\lstset{style=commands}
	\begin{lstlisting}[caption={Anlegen eines Users der Berechtigungen zum ausführen von 'tcpdump' erhält.}]
		elbrus@server:~$ sudo useradd aragog
	\end{lstlisting}
	
	\lstset{style=commands}
	\begin{lstlisting}[caption={Hinzufügen von User 'aragog' zu Gruppe 'elbrus'.}]
		elbrus@server:~$ sudo usermod -aG elbrus aragog
	\end{lstlisting}
	
	\lstset{style=commands}
	\begin{lstlisting}[caption={Zuweisen von 'tcpdump' zu der Gruppe 'aragog'.}]
		elbrus@server:~$ sudo chgrp aragog /usr/sbin/tcpdump
	\end{lstlisting}
	
	\lstset{style=commands}
	\begin{lstlisting}[caption={Ändern der Berechtigungen auf 'tcpdump'.}]
		elbrus@server:~$ sudo chmod 750 /usr/sbin/tcpdump
		elbrus@server:~$ sudo setcap cap_net_raw,cap_net_admin=eip \
		/usr/sbin/tcpdump
	\end{lstlisting}
	
	\lstset{style=commands}
	\begin{lstlisting}[caption={Wechseln des Owners \& der Berechtigung auf '/var/elbrus/capture/'}]
		elbrus@server:~$ sudo chown -R aragog:aragog /var/elbrus/capture/
		elbrus@server:~$ sudo chmod -R 770 /var/elbrus/capture/
		elbrus@server:~$ sudo chmod 777 /var/elbrus/capture/
	\end{lstlisting}
	
	\lstset{style=commands}
	\begin{lstlisting}caption={Ändern der Berechtigung auf '/var/elbrus/capture/install.sh'
		elbrus@server:~/var/elbrus$ sudo chmod 777 /var/elbrus/capture/install.sh
	\end{lstlisting}
	\newpage
	
	\subsection[file config]{Umgebung Konfigurieren}
	\subsubsection{1 - Mit Setup Script}
	
	\lstset{style=commands}
	\begin{lstlisting}[caption={Ausführen des 'install.sh' Scripts.}, breaklines=true,]
		elbrus@server:~/var/elbrus$ bash capture/install.sh
		Do you want to proceed with setup of the 'capture-device'? (y/n) y
		
		Where should the log be stored (dir) [/var/elbrus/shared/log]:
		Where is the elb-capture-postrotate.sh stored [/var/elbrus/capture/elb-capture-postrotate.sh]:
		Where is the shared config stored [/var/elbrus/shared/.config]:
		
		Should the log be stored at '/var/elbrus/shared/log' ?
		Is the 'elb-capture-postrotate.sh' stored at '/var/elbrus/capture/elb-capture-postrotate.sh' ?
		Is the shared config stored at '/var/elbrus/shared/.config' ? (y/n/exit) y
		#global
		SHAREDCONFIG=/var/elbrus/shared/.config
		
		#paths
		POSTROTATESCRIPT=/var/elbrus/capture/elb-capture-postrotate.sh
		LOGFILE=/var/elbrus/shared/log
		
		#settings
		TIMEPERCAPTURE=900
		MAXFILES=10
		INTERFACE=eth0
		Cleaning up...
		elbrus@server:~/var/elbrus$
	\end{lstlisting}
	
	\subsubsection{2 - Ohne Setup Script}
	
	\lstset{style=files}
	\begin{lstlisting}[caption={Anhand von '.env.example' eigene '.env' Datei anlegen.}, language=bash]
		#global
		SHAREDCONFIG=/var/elbrus/shared/.config
		
		#path
		POSTROTATESCRIPT=/var/elbrus/capture/elb-capture-postrotate.sh
		LOGFILEDIR=/var/elbrus/capture/capture"-$(date +"%Y-%U")".log
		
		#settings
		TIMEPERCAPTURE=900
		MAXFILES=10
		INTERFACE=eth0
	\end{lstlisting}
	\newpage
	
	\subsection[systemd service]{Der Systemd Service}
	
	\lstset{style=files}
	\begin{lstlisting}[caption={capture.service.example - Die Variable 'WorkingDirectory', Die Variable 'User' sowie die Variable 'ExecStopPost' anpassen.},language=bash ,keywords={WorkingDirectory, User, ExecStopPost}, keywordstyle=\color{red}, firstnumber=3]
		...
		#job is starting immediatly after the start action has been called
		Type=simple
		#the user to execute the script
		User=aragog
		#the working directory
		WorkingDirectory=/var/elbrus/capture
		#which script should be executed
		ExecStart=/bin/bash elb-capture.sh
		#when the script should restart
		Restart=on-failure
		#set the restart timeout
		RestartSec=5
		#which script should be executed when the service stops
		ExecStopPost=/bin/bash elb-capture-log.sh
		
		[Install]
		...
	\end{lstlisting}
	
	\lstset{style=commands}
	\begin{lstlisting}[caption={Kopieren des Serviceprogrammes.}]
		elbrus@server:~/var/elbrus$ sudo cp capture/capture.service.example\
		/etc/systemd/system/capture.service
	\end{lstlisting}
	
	\lstset{style=commands}
	\begin{lstlisting}[caption={Neuladen des 'systemctl' Deamons.}]
		elbrus@server:~/var/elbrus$ sudo systemctl daemon-reload
	\end{lstlisting}
	
	\lstset{style=commands}
	\begin{lstlisting}[caption={Aktivieren des Serviceprogrammes.}]
		elbrus@server:~/var/elbrus$ sudo systemctl enable capture.service
	\end{lstlisting}
	
	\lstset{style=commands}
	\begin{lstlisting}[caption={Starten des Serviceprogrammes.}]
		elbrus@server:~/var/elbrus$ sudo systemctl start capture.service
	\end{lstlisting}
	\newpage
	
	% --------- PACKET CAPTURE IMPORTER ---------
	\section{Packet Capture Importer}
	% --------- deployment guide for the importer ---------	
	\subsection[file config]{Umgebung Konfigurieren}
	\subsubsection{1 - Mit Setup Script}
	
	\lstset{style=commands}
	\begin{lstlisting}[caption={Ausführen des 'install.sh' Scripts.}, breaklines=true,]
		elbrus@server:~$ cd /var/elbrus
		elbrus@server:/var/elbrus$ bash database/importer/pcap-importer/\
		install.sh
		Do you want to proceed? (y/n) y
		
		Where is the shared config stored [/var/elbrus/shared/.config]: 
		Where should the 'pcap-importer' (dir) be stored [/var/elbrus/pcap-importer]: 
		
		Would you like to store the 'pcap-importer' at '/var/elbrus/pcap-importer' ?
		Is the shared config stored at '/var/elbrus/shared/.config' ? (y/n/exit) y
		Submodule 'importer/pcap-importer/pcap-analyzer' (https://github.com/rusticata/pcap-analyzer.git) registered for path 'importer/pcap-importer/pcap-analyzer'
		Cloning into '/var/elbrus/database/importer/pcap-importer/pcap-analyzer'...
		Submodule path 'importer/pcap-importer/pcap-analyzer': checked out '26abc0b0f4d9b2f0e6a72a62e694cd60ae6b6011'
		Start Building ... (this may take a while)
		<@\textcolor{codegreen}{Compiling}@> proc-macro2 v1.0.38
		<@\textcolor{codegreen}{Compiling}@> unicode-xid v0.2.3
		<@\textcolor{codegreen}{Compiling}@> syn v1.0.93
		...
		<@\textcolor{codegreen}{Compiling}@> libpcap-tools v0.1.0 (/var/elbrus/database/importer/pcap-importer/pcap-analyzer/libpcap-tools)
		<@\textcolor{codegreen}{Compiling}@> tokio-postgres v0.7.6
		<@\textcolor{codegreen}{Compiling}@> pcap-importer v0.1.0 (/var/elbrus/database/importer/pcap-importer)
		Finished release [optimized] target(s) in 1m 38s
		Cleaning up...
		elbrus@server:/var/elbrus$
	\end{lstlisting}
	\newpage
	
	\subsubsection{2 - Ohne Setup Script}	
	
	\lstset{style=files}
	\begin{lstlisting}[caption={Anhand von '.env.example' eigene '.env' Datei anlegen.}, language=bash]
		#global
		SHAREDCONFIG=/var/elbrus/shared/.config
	\end{lstlisting}
	
	\lstset{style=commands}
	\begin{lstlisting}[caption={Updaten der git Submodule.}]
		elbrus@server:/var/elbrus$ git -C database submodule update --init
	\end{lstlisting}
	
	\lstset{style=commands}
	\begin{lstlisting}[caption={Kompilieren des 'pcap-importers'.}]
		elbrus@server:/var/elbrus$ cargo build --release --manifest-path \
		database/importer/pcap-importer/Cargo.toml
	\end{lstlisting}
	
	\lstset{style=commands}
	\begin{lstlisting}[caption={Kopieren des 'pcap-importers' in ein eigenes Verzeichniss.}]
		elbrus@server:/var/elbrus$ mkdir -p /var/elbrus/pcap-importer
		elbrus@server:/var/elbrus$ mv database/importer/pcap-importer/target/\
		release/pcap-importer /var/elbrus/pcap-importer/pcap-importer
	\end{lstlisting}
	\newpage
	
	% --------- REPORT GENERATOR ---------
	\section{Report Generator}
	% --------- deployment guide for the report generator ---------		
	\subsection[file config]{Umgebung Konfigurieren}
	\subsubsection{1 - Mit Setup Script}
	
	\lstset{style=commands}
	\begin{lstlisting}[caption={Ausführen des 'install.sh' Scripts.}]
		elbrus@server:~$ cd /var/elbrus
		elbrus@server:/var/elbrus$ bash report-generator/install.sh
		Do you want to proceed with setup of the 'report-generator'? (y/n) y
		
		Where is the shared config stored [/var/elbrus/shared/.config]:
		
		Is the shared config stored at '/var/elbrus/shared/.config' ? (y/n/exit) y
		Install dependencies ...
		
		...
		
		elbrus@server:/var/elbrus$
	\end{lstlisting}
	
	\subsubsection{2 - Ohne Setup Script}
	
	\lstset{style=files}
	\begin{lstlisting}[caption={Anhand von '.env.example' eigene '.env' Datei anlegen.}, language=bash]
		#global
		SHAREDCONFIG=/var/elbrus/shared/.config
	\end{lstlisting}
	
	\lstset{style=commands}
	\begin{lstlisting}[caption={Installieren von fehlenden python3 Packages.}]
		elbrus@server:/var/elbrus$ pip3 install -r \
		report-generator/requirements.txt
	\end{lstlisting}
	\newpage
	
	\subsection[systemd service]{Der Systemd Service}
	
	\lstset{style=files}
	\begin{lstlisting}[caption={report-generator.service.example - Die Variable 'WorkingDirectory' sowie die Variable 'User' anpassen.},language=bash ,keywords={WorkingDirectory, User}, keywordstyle=\color{red}, firstnumber=5]
		...
		#job is starting immediatly after the start action has been called
		Type=simple
		#the user to execute the script
		User=elbrus
		#the working directory
		WorkingDirectory=/var/elbrus/report-generator/src/
		#which script should be executed
		ExecStart=python3 main.py
		...
	\end{lstlisting}
	
	\lstset{style=commands}
	\begin{lstlisting}[caption={Kopieren des Serviceprogrammes.}]
		elbrus@server:/var/elbrus$ sudo cp report-generator/src/report-generator.service\
		.example /etc/systemd/system/report-generator.service
	\end{lstlisting}
	
	\lstset{style=commands}
	\begin{lstlisting}[caption={Kopieren des Zeitplanungsprogrammes.}]
		elbrus@server:/var/elbrus$ sudo cp report-generator/src/report-generator-schedule.timer\
		.example /etc/systemd/system/report-generator-schedule.timer
	\end{lstlisting}
	
	\lstset{style=commands}
	\begin{lstlisting}[caption={Neuladen des 'systemctl' Deamons.}]
		elbrus@server:/var/elbrus$ sudo systemctl daemon-reload
	\end{lstlisting}
	
	\lstset{style=commands}
	\begin{lstlisting}[caption={Aktivieren des Serviceprogrammes.}]
		elbrus@server:/var/elbrus$ sudo systemctl enable report-generator.service
	\end{lstlisting}
	
	\lstset{style=commands}
	\begin{lstlisting}[caption={Aktivieren des Zeitplanungsprogrammes.}]
		elbrus@server:/var/elbrus$ sudo systemctl enable report-generator-schedule.timer
	\end{lstlisting}
	
	\lstset{style=commands}
	\begin{lstlisting}[caption={Starten des Zeitplanungsprogrammes.}]
		elbrus@server:/var/elbrus$ sudo systemctl start report-generator-schedule.timer
	\end{lstlisting}
	\newpage
	
	
	% --------- SNMP MANAGER ---------
	\section{SNMP Manager}
	% --------- deployment guide for the snmp manager ---------		
	\subsection{1 - Mit Setup Script}
	
	\lstset{style=commands}
	\begin{lstlisting}[caption={Ausführen des 'install.sh' Scripts.}]
		elbrus@server:~$ cd /var/elbrus
		elbrus@server:/var/elbrus$ sudo bash snmp-manager/src/install.sh 
		Do you want to proceed with setup of the 'snmp-manager'? (y/n) y
		
		Where should the log be stored (dir) [/var/elbrus/shared/log]: 
		Where should the 'snmp-manager' be stored [/var/elbrus/snmp-manager]:
		Where is the shared config stored [/var/elbrus/shared/.config]: 
		
		Should the log be stored at '/var/elbrus/shared/log' ?
		Should the 'snmp-manager' be stored at '/var/elbrus/snmp-manager'?
		Is the shared config stored at '/var/elbrus/shared/.config' ? (y/n/exit) y
		
		<@\textcolor{codegreen}{Success!}@> .env file was created
		<@\textcolor{codegreen}{Success!}@> systemd service was automatically deployed,
		<@\textcolor{codegreen}{Success!}@> installed requirements
		
		elbrus@server:/var/elbrus$
	\end{lstlisting}
	
	\subsection{2 - Ohne Setup Script}
	
	\subsubsection[file config]{Umgebung Konfigurieren}
	\lstset{style=files}
	\begin{lstlisting}[caption={Anhand von '.env.example' eigene '.env' Datei anlegen.}, language=bash, keywords={CONFIGPATH, MAINPATH}, keywordstyle=\color{red}]
		#global
		SHAREDCONFIG=/var/elbrus/shared/.config
		
		#paths
		LOGFILEDIR=/var/elbrus/shared/log
	\end{lstlisting}
	\newpage
	
	\subsubsection[systemd service]{Der Systemd Service}
	
	\lstset{style=files}
	\begin{lstlisting}[caption={snmp-manager.service.example - Die Variable 'WorkingDirectory' sowie die Variable 'User' anpassen.},language=bash ,keywords={WorkingDirectory, User}, keywordstyle=\color{red}, firstnumber=5]
		...
		#job is starting immediatly after the start action has been called
		Type=simple
		#the user to execute the script
		User=elbrus
		#the working directory
		WorkingDirectory=/var/elbrus/snmp-manager/src
		#which script should be executed
		ExecStart=/bin/bash elb-snmp-manager.sh
		...
	\end{lstlisting}
	
	\lstset{style=commands}
	\begin{lstlisting}[caption={Kopieren des Serviceprogrammes.}]
		elbrus@server:/var/elbrus$ sudo cp snmp-manager/src/snmp-manager.service\
		.example /etc/systemd/system/snmp-manager.service
	\end{lstlisting}
	
	\lstset{style=commands}
	\begin{lstlisting}[caption={Kopieren des Zeitplanungsprogrammes.}]
		elbrus@server:/var/elbrus$ sudo cp snmp-manager/src/snmp-manager-schedule.timer\
		.example /etc/systemd/system/snmp-manager-schedule.timer
	\end{lstlisting}
	
	\lstset{style=commands}
	\begin{lstlisting}[caption={Neuladen des 'systemctl' Deamons.}]
		elbrus@server:/var/elbrus$ sudo systemctl daemon-reload
	\end{lstlisting}
	
	\lstset{style=commands}
	\begin{lstlisting}[caption={Aktivieren des Serviceprogrammes.}]
		elbrus@server:/var/elbrus$ sudo systemctl enable snmp-manager.service
	\end{lstlisting}
	
	\lstset{style=commands}
	\begin{lstlisting}[caption={Aktivieren des Zeitplanungsprogrammes.}]
		elbrus@server:/var/elbrus$ sudo systemctl enable snmp-manager-schedule.timer
	\end{lstlisting}
	
	\lstset{style=commands}
	\begin{lstlisting}[caption={Starten des Zeitplanungsprogrammes.}]
		elbrus@server:/var/elbrus$ sudo systemctl start snmp-manager-schedule.timer
	\end{lstlisting}
	\newpage
	
	% --------- SSH MANAGER ---------
	\section{SSH Manager}
	% --------- deployment guide for the ssh manager ---------		
	\subsection[file config]{Umgebung Konfigurieren}
	\subsubsection{1 - Mit Setup Script}
	
	\lstset{style=commands}
	\begin{lstlisting}[caption={Ausführen des 'install.sh' Scripts.}]
		elbrus@server:~$ cd /var/elbrus
		elbrus@server:~$ bash ssh-manager/src/install.sh
		Do you want to proceed with the setup of the 'ssh-manager'? (y/n) y
		we will proceed
		
		Where do you want the ssh config replies to be stored (dir) [/var/elbrus/
		shared/ssh-configs]
		Where is the 'main.py' file stored [/var/elbrus/ssh-manager/src/main.py]
		Where is the shared config stored [/var/elbrus/shared/.config]
		
		Do you want to store the config files at '/var/elbrus/shared/ssh-configs'?
		Is your 'main.py' stored at '/var/elbrus/ssh-manager/src/main.py'?
		Is the shared config stored at '/var/elbrus/shared/.config'? (y/n/exit) y
		we will proceed
		
		#global
		SHAREDCONFIG=/var/elbrus/shared/.config
		
		#values regarding the jumpserver:
		#IP, PORT and USER values must be set!
		#depending on the usage you can set either:
		#   -PASS and KEYFILE: the keyfile is used, the pass is interpreted as the
		passphrase
		#   -only KEYFILE: the keyfile is used
		#   -only PASS: the password is used as is regular credentials
		JUMPSERVER_IP=
		JUMPSERVER_PORT=
		JUMPSERVER_USER=
		JUMPSERVER_PASS=
		SSH_KEYFILE=
		
		#paths
		CONFIGPATH=/var/elbrus/shared/ssh-configs
		MAINPATH=/var/elbrus/ssh-manager/src/main.py
		
		Do you want to run the setup script? (y/n/exit) y
		...
		Initialized empty Git repository in /var/elbrus/shared/ssh-configs/.git/
		info: created config folder!
		Install dependencies ...
		...
		Cleaning up...
		elbrus@server:/ssh-manager$
	\end{lstlisting}
	\newpage
	
	\lstset{style=files}
	\begin{lstlisting}[caption={Ergänzen der fehlenden Werten in '.env'.}, language=bash]
		#values regarding the jumpserver:
		#IP, PORT and USER values must be set!
		#depending on the usage you can set either:
		#   -PASS and KEYFILE: the keyfile is used, the pass is interpreted as the passphrase
		#   -only KEYFILE: the keyfile is used
		#   -only PASS: the password is used as is regular credentials
		JUMPSERVER_IP=
		JUMPSERVER_PORT=
		JUMPSERVER_USER=
		JUMPSERVER_PASS=
		SSH_KEYFILE=
	\end{lstlisting}
	
	\subsubsection{2 - Ohne Setup Script}
	
	\lstset{style=files}
	\begin{lstlisting}[caption={Anhand von '.env.example' eigene '.env' Datei anlegen.}, language=bash]
		#global
		SHAREDCONFIG=/var/elbrus/shared/.config
		
		#values regarding the jumpserver:
		#IP, PORT and USER values must be set!
		#depending on the usage you can set either:
		#   -PASS and KEYFILE: the keyfile is used, the pass is interpreted as the passphrase
		#   -only KEYFILE: the keyfile is used
		#   -only PASS: the password is used as is regular credentials
		JUMPSERVER_IP=
		JUMPSERVER_PORT=
		JUMPSERVER_USER=
		JUMPSERVER_PASS=
		SSH_KEYFILE=
		
		#paths
		CONFIGPATH=/var/elbrus/shared/ssh-configs
		MAINPATH=/var/elbrus/ssh-manager/src/main.py
	\end{lstlisting}
	
	\lstset{style=commands}	
	\begin{lstlisting}[caption={Ausführen des Scripts zur Initialisierung des VCS Verzeichnisses.}]
		elbrus@server:/var/elbrus/ssh-manager$ bash src/setup.sh
	\end{lstlisting}
	
	\lstset{style=commands}
	\begin{lstlisting}[caption={Installieren von fehlenden python3 Packages.}]
		elbrus@server:/var/elbrus/ssh-manager$ pip3 install -r requirements.txt
	\end{lstlisting}
	\newpage
	
	\subsection[systemd service]{Der Systemd Service}
	
	\lstset{style=files}
	\begin{lstlisting}[caption={ssh-manager.service.example - Die Variable 'WorkingDirectory' sowie die Variable 'User' anpassen.},language=bash ,keywords={WorkingDirectory, User}, keywordstyle=\color{red}, firstnumber=5]
		...
		#job is starting immediatly after the start action has been called
		Type=simple
		#the user to execute the script
		User=elbrus
		#the working directory
		WorkingDirectory=/var/elbrus/ssh-manager/src/
		#which script should be executed
		ExecStart=/bin/bash routine.sh
		...
	\end{lstlisting}
	
	\lstset{style=commands}
	\begin{lstlisting}[caption={Kopieren des Serviceprogrammes.}]
		elbrus@server:/var/elbrus$ sudo cp ssh-manager/src/ssh-manager.service.example \
		/etc/systemd/system/ssh-manager.service
	\end{lstlisting}
	
	\lstset{style=commands}
	\begin{lstlisting}[caption={Kopieren des Zeitplanungsprogrammes.}]
		elbrus@server:/var/elbrus$ sudo cp ssh-manager/src/ssh-manager-schedule.timer.example \
		/etc/systemd/system/ssh-manager-schedule.timer
	\end{lstlisting}
	
	\lstset{style=commands}
	\begin{lstlisting}[caption={Neuladen des 'systemctl' Deamons.}]
		elbrus@server:/var/elbrus$ sudo systemctl daemon-reload
	\end{lstlisting}
	
	\lstset{style=commands}
	\begin{lstlisting}[caption={Aktivieren des Serviceprogrammes.}]
		elbrus@server:/var/elbrus$ sudo systemctl enable ssh-manager.service
	\end{lstlisting}
	
	\lstset{style=commands}
	\begin{lstlisting}[caption={Aktivieren des Zeitplanungsprogrammes.}]
		elbrus@server:/var/elbrus$ sudo systemctl enable ssh-manager-schedule.timer
	\end{lstlisting}
	
	\lstset{style=commands}
	\begin{lstlisting}[caption={Starten des Zeitplanungsprogrammes.}]
		elbrus@server:/var/elbrus$ sudo systemctl start ssh-manager-schedule.timer
	\end{lstlisting}
	\newpage
	
	% --------- UPTIME MONITOR ---------
	\section{Uptime Monitor}
	% --------- deployment guide for the uptime monitor ---------	
	\subsection[file config]{Umgebung Konfigurieren}
	\subsubsection{1 - Mit Setup Script}
	
	\lstset{style=commands}
	\begin{lstlisting}[caption={Ausführen des 'install.sh' Scripts.}]
		elbrus@server:~$ cd /var/elbrus
		elbrus@server:~$ v
		Do you want to proceed with the setup of the 'uptime-monitor'? (y/n) y
		
		Where is the shared config stored [/var/elbrus/shared/.config]:
		Where should the 'uptime-monitor' be stored [/var/elbrus/uptime-monitor]:
		
		Is the shared config stored at '/var/elbrus/shared/.config'?
		Should the 'uptime-monitor be stored at '/var/elbrus/uptime-monitor'? (y/n/exit) y
		
		<@\textcolor{codegreen}{Success!}@> .env file was created
		<@\textcolor{codegreen}{Success!}@> systemd service was automatically deployed,
		elbrus@server:~$
	\end{lstlisting}
	
	\subsubsection{2 - Ohne Setup Script}
	
	\lstset{style=files}
	\begin{lstlisting}[caption={Anhand von '.env.example' eigene '.env' Datei anlegen.}, language=bash]
		#global
		SHAREDCONFIG=/var/elbrus/shared/.config
		
		#config
		# Initial pings to see if device is alive
		INITIALPING=1
		# Pings to get the availability statistic
		STATISTICPING=10
	\end{lstlisting}
	\newpage
	
	\subsection[systemd service]{Der Systemd Service}
	
	\lstset{style=files}
	\begin{lstlisting}[caption={uptime\_monitor.service.example - Die Variable 'WorkingDirectory' sowie die Variable 'User' anpassen.},language=bash ,keywords={WorkingDirectory, User}, keywordstyle=\color{red}, firstnumber=5]
		...
		#job is starting immediatly after the start action has been called
		Type=simple
		#the user to execute the script
		User=elbrus
		#the working directory
		WorkingDirectory=/var/elbrus/uptime_monitor
		#which script should be executed
		ExecStart=/bin/bash uptime_monitor.sh
		...
	\end{lstlisting}
	
	\lstset{style=commands}
	\begin{lstlisting}[caption={Kopieren des Serviceprogrammes.}]
		elbrus@server:/var/elbrus$ sudo cp uptime_monitor/uptime_monitor.service.example \
		/etc/systemd/system/uptime_monitor.service
	\end{lstlisting}
	
	\lstset{style=commands}
	\begin{lstlisting}[caption={Kopieren des Zeitplanungsprogrammes.}]
		elbrus@server:/var/elbrus$ sudo cp uptime_monitor/uptime_monitor-schedule.timer.example \
		/etc/systemd/system/uptime_monitor-schedule.timer
	\end{lstlisting}
	
	\lstset{style=commands}
	\begin{lstlisting}[caption={Neuladen des 'systemctl' Deamons.}]
		elbrus@server:/var/elbrus$ sudo systemctl daemon-reload
	\end{lstlisting}
	
	\lstset{style=commands}
	\begin{lstlisting}[caption={Aktivieren des Serviceprogrammes.}]
		elbrus@server:/var/elbrus$ sudo systemctl enable uptime_monitor.service
	\end{lstlisting}
	
	\lstset{style=commands}
	\begin{lstlisting}[caption={Aktivieren des Zeitplanungsprogrammes.}]
		elbrus@server:/var/elbrus$ sudo systemctl enable uptime_monitor-schedule.timer
	\end{lstlisting}
	
	\lstset{style=commands}
	\begin{lstlisting}[caption={Starten des Zeitplanungsprogrammes.}]
		elbrus@server:/var/elbrus$ sudo systemctl start uptime_monitor-schedule.timer
	\end{lstlisting}
	\newpage
	
	% --------- GEO SESSION FINDER ---------
	\section{Geo Session finder}
	% --------- deployment guide for the geo session finder ---------	
	\subsection{1 - Mit Setup Script}
	
	\lstset{style=commands}
	\begin{lstlisting}[caption={Ausführen des 'install.sh' Scripts.}]
		elbrus@server:~$ cd /var/elbrus
		elbrus@server:~$ sudo bash geo-session-finder/src/install.sh
		Do you want to proceed with setup of the 'geo session finder'? (y/n) y
		
		Where should the log be stored (dir) [/var/elbrus/shared/log]:
		Where should the 'geo-session-finder' be stored [/var/elbrus/geo-session-finder]:
		Where is the shared config stored [/var/elbrus/shared/.config]:
		
		Should the log be stored at '/var/elbrus/shared/log' ?
		Should the 'geo session finder' be stored at '/var/elbrus/geo-session-finder'?
		Is the shared config stored at '/var/elbrus/shared/.config' ? (y/n/exit) y
		
		<@\textcolor{codegreen}{Success!}@> .env file was created
		<@\textcolor{codegreen}{Success!}@> systemd service was automatically deployed,
		
		Installing dependencies ...
		
		...
		
		elbrus@server:~$	
	\end{lstlisting}

	\subsection{2 - Ohne Setup Script}

	\lstset{style=commands}
	\begin{lstlisting}[caption={Installieren der Abhängigkeiten.}]
		elbrus@server:~$ pip3 install -r geo_session_finder/requirements.txt
	\end{lstlisting}

	\subsubsection[file config]{Umgebung Konfigurieren}
	\lstset{style=files}
	\begin{lstlisting}[caption={Anhand von '.env.example' eigene '.env' Datei anlegen.}, language=bash]
		#global
		SHAREDCONFIG=/var/elbrus/shared/.config
		
		#paths
		LOGFILEDIR=/var/elbrus/shared/log
	\end{lstlisting}
	\newpage
	
	\subsubsection[systemd service]{Der Systemd Service}
	
	\lstset{style=files}
	\begin{lstlisting}[caption={geo-session-finder.service.example - Die Variable 'WorkingDirectory' sowie die Variable 'User' anpassen.},language=bash ,keywords={WorkingDirectory, User}, keywordstyle=\color{red}, firstnumber=5]
		...
		#job is starting immediatly after the start action has been called
		Type=simple
		#the user to execute the script
		User=elbrus
		#the working directory
		WorkingDirectory=/var/elbrus/geo_session_finder/src
		#which script should be executed
		ExecStart=/bin/bash elb-geo-session-finder.sh
		...
	\end{lstlisting}
	
	\lstset{style=commands}
	\begin{lstlisting}[caption={Kopieren des Serviceprogrammes.}]
		elbrus@server:/var/elbrus$ sudo cp geo_session_finder/src/geo-session-finder\
		.service.example /etc/systemd/system/geo-session-finder.service
	\end{lstlisting}
	
	\lstset{style=commands}
	\begin{lstlisting}[caption={Kopieren des Zeitplanungsprogrammes.}]
		elbrus@server:/var/elbrus$ sudo cp geo_session_finder/src/geo-session-finder-schedule\
		.timer.example /etc/systemd/system/geo-session-finder-schedule.timer
	\end{lstlisting}
	
	\lstset{style=commands}
	\begin{lstlisting}[caption={Neuladen des 'systemctl' Deamons.}]
		elbrus@server:/var/elbrus$ sudo systemctl daemon-reload
	\end{lstlisting}
	
	\lstset{style=commands}
	\begin{lstlisting}[caption={Aktivieren des Serviceprogrammes.}]
		elbrus@server:/var/elbrus$ sudo systemctl enable geo-session-finder.service
	\end{lstlisting}
	
	\lstset{style=commands}
	\begin{lstlisting}[caption={Aktivieren des Zeitplanungsprogrammes.}]
		elbrus@server:/var/elbrus$ sudo systemctl enable geo-session-finder-schedule.timer
	\end{lstlisting}
	
	\lstset{style=commands}
	\begin{lstlisting}[caption={Starten des Zeitplanungsprogrammes.}]
		elbrus@server:/var/elbrus$ sudo systemctl start geo-session-finder-schedule.timer
	\end{lstlisting}
	\newpage
	
	% --------- office365 ---------
	\section{office365}
	% --------- deployment guide for the office 365 get endpoint ---------
	\subsection[file config]{Umgebung Konfigurieren}
	\subsubsection{1 - Mit Setup Script}
	
	\lstset{style=commands}
	\begin{lstlisting}[caption={Ausführen des 'install.sh' Scripts.}]
		elbrus@server:~$ cd /var/elbrus
		elbrus@server:~$ sudo bash office365-analyzer/src/install.sh
		Do you want to proceed with setup of the 'office365-analyzer'? (y/n) y
		
		Where should the 'office365-analyzer' be stored (dir) [/var/elbrus/office365-analyzer]:
		Where should the log be stored (dir) [/var/elbrus/shared/log]:
		Where is the shared config stored [/var/elbrus/shared/.config]:
		
		Should the 'office365-analyzer' be stored at '/var/elbrus/office365-analyzer' ?
		Should the log be stored at '/var/elbrus/shared/log' ?
		Is the shared config stored at '/var/elbrus/shared/.config' ? (y/n/exit) y
		
		<@\textcolor{codegreen}{Success!}@> .env file was created
		<@\textcolor{codegreen}{Success!}@> systemd service was automatically deployed,
		
		info: installing dependencies.
		
		...
		
		info: installed dependencies.
		
		elbrus@server:~$	
	\end{lstlisting}
	
	\subsubsection{2 - Ohne Setup Script}
	
	\lstset{style=files}
	\begin{lstlisting}[caption={Anhand von '.env.example' eigene '.env' Datei anlegen.}, language=bash]
		#global
		SHAREDCONFIG=/var/elbrus/shared/.config
		
		#paths
		LOGFILEDIR=/var/elbrus/shared/log
		
		#ms url
		MS_URL=https://endpoints.office.com/endpoints/worldwide?clientrequestid=b10c5ed1-bad1-445f-b386-b919946339a7
	\end{lstlisting}
	\newpage

	\subsection[systemd service]{Der Systemd Service}
	
	\lstset{style=files}
	\begin{lstlisting}[caption={uptime\_monitor.service.example - Die Variable 'WorkingDirectory' sowie die Variable 'User' anpassen.},language=bash ,keywords={WorkingDirectory, User}, keywordstyle=\color{red}, firstnumber=4]
		...
		#job is starting immediatly after the start action has been called
		Type=simple
		#the user to execute the script
		User=elbrus
		#the working directory
		WorkingDirectory=/var/elbrus/office365-analyzer/src
		#which script should be executed
		ExecStart=/bin/bash elb-office365-get-endpoints.sh
		...
	\end{lstlisting}
	
	\lstset{style=commands}
	\begin{lstlisting}[caption={Kopieren des Serviceprogrammes.}]
		elbrus@server:/var/elbrus$ sudo cp office365-analyzer/src/office365-get-endpoints\
		.service.example /etc/systemd/system/office365-get-endpoints.service
	\end{lstlisting}
	
	\lstset{style=commands}
	\begin{lstlisting}[caption={Kopieren des Zeitplanungsprogrammes.}]
		elbrus@server:/var/elbrus$ sudo cp office365-analyzer/src/office365-get-endpoints\
		-schedule.timer.example /etc/systemd/system/office365-get-endpoints-schedule.timer
	\end{lstlisting}
	
	\lstset{style=commands}
	\begin{lstlisting}[caption={Neuladen des 'systemctl' Deamons.}]
		elbrus@server:/var/elbrus$ sudo systemctl daemon-reload
	\end{lstlisting}
	
	\lstset{style=commands}
	\begin{lstlisting}[caption={Aktivieren des Serviceprogrammes.}]
		elbrus@server:/var/elbrus$ sudo systemctl enable office365-get-endpoints.service
	\end{lstlisting}
	
	\lstset{style=commands}
	\begin{lstlisting}[caption={Aktivieren des Zeitplanungsprogrammes.}]
		elbrus@server:/var/elbrus$ sudo systemctl enable office365-get-endpoints-schedule.timer
	\end{lstlisting}
	
	\lstset{style=commands}
	\begin{lstlisting}[caption={Starten des Serviceprogrammes \& Zeitplanungsprogrammes.}]
		elbrus@server:/var/elbrus$ sudo systemctl start office365-get-endpoints.service
		elbrus@server:/var/elbrus$ sudo systemctl start office365-get-endpoints-schedule.timer
	\end{lstlisting}
	\newpage
	
	% --------- API ---------
	\section{API}
	% --------- deployment guide for the API ---------		
	\subsection{Voraussetzungen}
	
	\lstset{style=commands}
	\begin{lstlisting}[caption={Installieren von 'pm2'.}]
		elbrus@server:~/var/elbrus$ sudo npm install -g pm2
	\end{lstlisting}
	
	\lstset{style=commands}
	\begin{lstlisting}[caption={Nachinstallieren der Abhängigkeiten.}]
		elbrus@server:~/var/elbrus$ cd api
		elbrus@server:~/var/elbrus/api$ sudo npm install
	\end{lstlisting}
	\newpage
	
	\subsection[file config]{Umgebung Konfigurieren}
	
	\lstset{style=files}
	\begin{lstlisting}[caption={Anhand von '.env.example' eigene '.env' Datei anlegen.}, language=bash]
		# Application Name
		APP_NAME=Elbrus-API
		
		# Port number
		PORT=3000
		
		# BASE URL
		BASE=https://localhost:3000
		
		# URL of DB
		DB_USER=
		DB_HOST=
		DB_DATABASE=
		DB_PASSWORD=
		DB_PORT=
		
		# JWT
		JWT_SECRET=thisisasamplesecret
		JWT_ACCESS_EXPIRATION_MINUTES=30
		JWT_REFRESH_EXPIRATION_DAYS=30
		
		# SMTP configuration options for the email service
		SMTP_HOST=
		SMTP_PORT=
		SMTP_USERNAME=
		SMTP_PASSWORD=
		EMAIL_FROM=
		EMAIL_NAME=
	\end{lstlisting}
	
	\subsection{Inbetriebnahme}
	
	\lstset{style=commands}
	\begin{lstlisting}[caption={Starten der API.}]
		elbrus@server:~/var/elbrus/api$ pm2 start ecosystem.config.json
	\end{lstlisting}
	Die API läuft in folge automatisch im Hintergrund.
	\newpage
	
	\begin{enumerate}
		\item \textbf{APP\_NAME} wird rein als beschreibender Name genutzt und kann so belassen werden.
		\item \textbf{PORT} beschreibt den TCP Port auf dem die Applikation laufen soll.
		\item \textbf{BASE} ist der Wert der Basis URL auf welche zugegriffen wird. Hier muss der Port auch angegeben werden!
		\item \textbf{DB\_USER} ist der benutzername des DBMS Benutzers, über welchen der Zugriff auf die Datenbank läuft.
		\item \textbf{DB\_HOST} ist der hostname/ip-adresse des Servers welcher die Datenbank hostet.
		\item \textbf{DB\_DATABASE} beschreibt den Namen der Datenbank selber.
		\item \textbf{DB\_PASSWORD} ist das Passwort des DBMS Benutzers, über welchen der Zugriff auf die Datenbank läuft.
		\item \textbf{DB\_PORT} ist der TCP Port des Servers welcher die Datenbank hostet.
		\item \textbf{JWT\_SECRET} ist das Passwort mit dem alle JWT Tokens ausgestellt werden.
		\item \textbf{JWT\_ACCESS\_EXPIRATION\_MINUTES} gibt die Dauer der Gültigkeit eines Access-Tokens an (in Minuten)
		\item \textbf{JWT\_REFRESH\_EXPIRATION\_DAYS} gibt die Dauer der Gültigkeit eines Refresh-Tokens an (in Tagen)
		\item \textbf{SMTP\_HOST} ist der hostname/ip-adresse des EMail Servers
		\item \textbf{SMTP\_PORT} ist der TCP Port des EMail Servers für SMTP
		\item \textbf{SMTP\_USERNAME} ist der username des Benutzers zum einloggen in den EMail Account
		\item \textbf{SMTP\_PASSWORD} ist das passwort des Benutzers zum einloggen in den EMail Account
		\item \textbf{EMAIL\_FROM} gibt die Email adresse an, von welcher gesendet werden soll.
		\item \textbf{EMAIL\_NAME} gibt den Namen an, welcher dem Empfänger angezeigt werden soll.
	\end{enumerate}
	\newpage
	
	% --------- Webinterface ---------
	\section{Webinterface}
	\newpage
\end{document}