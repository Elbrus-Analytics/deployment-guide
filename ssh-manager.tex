% --------- deployment guide for the ssh manager ---------		
	\subsection{1 - Mit Setup Script}
	
	\lstset{style=commands}
	\begin{lstlisting}[caption={Ausführen des 'install.sh' Scripts.}]
		elbrus@server:~$ cd /var/elbrus
		elbrus@server:/var/elbrus$ sudo bash ssh-manager/src/install.sh 
		Do you want to proceed with the setup of the 'ssh-manager'? (y/n) y
		
		Where do you want the ssh config replies to be stored (dir) 
		[/var/elbrus/shared/ssh-configs]:
		Where should the 'ssh-manager' be stored [/var/elbrus/ssh-manager]: 
		Where is the shared config stored [/var/elbrus/shared/.config]:
		
		Do you want to store the config files at '/var/elbrus/shared/ssh-configs'?
		Do you want to store the 'ssh-manager' at '/var/elbrus/ssh-manager'?
		Is the shared config stored at '/var/elbrus/shared/.config'? (y/n/exit) y
		
		info: Be aware if you skip you will have to enter the values manually in
		 '/var/elbrus/ssh-manager/.env'
		Do you want to configure the jumphost settings (y/skip/exit): y
		info: While configuring the credentials you can choose if you want to configure
		) only the ssh-key   (leave password blank)   <-- only ssh keyfile is used to connect
		) only the password  (leave ssh-key blank)    <-- only password is used to connect
		) both                                        <-- keyfile is used with password as 
		 passphrase
		
		
		What is your Jumpserver IP (e.g. 1.2.3.4): 1.2.3.4
		On which Port is the Jumpserver listening (e.g. 22): 22
		What is your Jumpserver Username (e.g. elbrus): elburs
		What is your Jumpserver Password (e.g. ******): elbrus
		Where is your ssh-key located (e.g. /var/elbrus/shared/elbrus_key):
		Please check the following settings:
		HOST: 1.2.3.4
		PORT: 22
		USER: elburs
		PASSWORD: elbrus
		SSH-KEY:
		are these your settings? (y/n/exit) y
		Success! .env file was created, Please fill in the unfilled values.
		Success! systemd service was automatically deployed,
		info: installing dependencies.
		info: installed dependencies.
		
		
		<@\textcolor{codegreen}{Success!}@> .env file was created
		<@\textcolor{codegreen}{Success!}@> systemd service was automatically deployed,
		
		Do you want to run the setup script? (y/n/exit) y
		
		...
		
		info: created config folder!
		Installing dependencies ...
	
		...
		
		elbrus@server:/var/elbrus$
	\end{lstlisting}
	\newpage
	
	\subsection{2 - Ohne Setup Script}
	\subsubsection[file config]{Umgebung Konfigurieren}
	\lstset{style=files}
	\begin{lstlisting}[caption={Anhand von '.env.example' eigene '.env' Datei anlegen.}, language=bash]
		#global
		SHAREDCONFIG=/var/elbrus/shared/.config
		
		#values regarding the jumpserver:
		#IP, PORT and USER values must be set!
		#depending on the usage you can set either:
		#   -PASS and KEYFILE: the keyfile is used, the pass is interpreted as the passphrase
		#   -only KEYFILE: the keyfile is used
		#   -only PASS: the password is used as is regular credentials
		JUMPSERVER_IP=
		JUMPSERVER_PORT=
		JUMPSERVER_USER=
		JUMPSERVER_PASS=
		SSH_KEYFILE=
		
		#paths
		CONFIGPATH=/var/elbrus/shared/ssh-configs
		MAINPATH=/var/elbrus/ssh-manager/src/main.py
	\end{lstlisting}
	
	\lstset{style=commands}	
	\begin{lstlisting}[caption={Ausführen des Scripts zur Initialisierung des VCS Verzeichnisses.}]
		elbrus@server:/var/elbrus/ssh-manager$ bash src/setup.sh
	\end{lstlisting}
	
	\lstset{style=commands}
	\begin{lstlisting}[caption={Installieren von fehlenden python3 Packages.}]
		elbrus@server:/var/elbrus/ssh-manager$ pip3 install -r requirements.txt
	\end{lstlisting}
	\newpage
	
	\subsubsection[systemd service]{Der Systemd Service}
	
	\lstset{style=files}
	\begin{lstlisting}[caption={ssh-manager.service.example - Die Variable 'WorkingDirectory' sowie die Variable 'User' anpassen.},language=bash ,keywords={WorkingDirectory, User}, keywordstyle=\color{red}, firstnumber=5]
		...
		#job is starting immediatly after the start action has been called
		Type=simple
		#the user to execute the script
		User=elbrus
		#the working directory
		WorkingDirectory=/var/elbrus/ssh-manager/src/
		#which script should be executed
		ExecStart=/bin/bash routine.sh
		...
	\end{lstlisting}
	
	\lstset{style=commands}
	\begin{lstlisting}[caption={Kopieren des Serviceprogrammes.}]
		elbrus@server:/var/elbrus$ sudo cp ssh-manager/src/ssh-manager.service.example \
		/etc/systemd/system/ssh-manager.service
	\end{lstlisting}
	
	\lstset{style=commands}
	\begin{lstlisting}[caption={Kopieren des Zeitplanungsprogrammes.}]
		elbrus@server:/var/elbrus$ sudo cp ssh-manager/src/ssh-manager-schedule.timer.example \
		/etc/systemd/system/ssh-manager-schedule.timer
	\end{lstlisting}
	
	\lstset{style=commands}
	\begin{lstlisting}[caption={Neuladen des 'systemctl' Deamons.}]
		elbrus@server:/var/elbrus$ sudo systemctl daemon-reload
	\end{lstlisting}
	
	\lstset{style=commands}
	\begin{lstlisting}[caption={Aktivieren des Serviceprogrammes.}]
		elbrus@server:/var/elbrus$ sudo systemctl enable ssh-manager.service
	\end{lstlisting}
	
	\lstset{style=commands}
	\begin{lstlisting}[caption={Aktivieren des Zeitplanungsprogrammes.}]
		elbrus@server:/var/elbrus$ sudo systemctl enable ssh-manager-schedule.timer
	\end{lstlisting}
	
	\lstset{style=commands}
	\begin{lstlisting}[caption={Starten des Zeitplanungsprogrammes.}]
		elbrus@server:/var/elbrus$ sudo systemctl start ssh-manager-schedule.timer
	\end{lstlisting}