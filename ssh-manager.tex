% --------- deployment guide for the ssh manager ---------		
	\subsection[file config]{Umgebung Konfigurieren}
	\subsubsection{1 - Mit Setup Script}
	
	\lstset{style=commands}
	\begin{lstlisting}[caption={Ausführen des 'install.sh' Scripts.}]
		elbrus@server:~$ cd /var/elbrus
		elbrus@server:~$ bash ssh-manager/src/install.sh
		Do you want to proceed with the setup of the 'ssh-manager'? (y/n) y
		we will proceed
		
		Where do you want the ssh config replies to be stored (dir) [/var/elbrus/
		shared/ssh-configs]
		Where is the 'main.py' file stored [/var/elbrus/ssh-manager/src/main.py]
		Where is the shared config stored [/var/elbrus/shared/.config]
		
		Do you want to store the config files at '/var/elbrus/shared/ssh-configs'?
		Is your 'main.py' stored at '/var/elbrus/ssh-manager/src/main.py'?
		Is the shared config stored at '/var/elbrus/shared/.config'? (y/n/exit) y
		we will proceed
		
		#global
		SHAREDCONFIG=/var/elbrus/shared/.config
		
		#values regarding the jumpserver:
		#IP, PORT and USER values must be set!
		#depending on the usage you can set either:
		#   -PASS and KEYFILE: the keyfile is used, the pass is interpreted as the
		passphrase
		#   -only KEYFILE: the keyfile is used
		#   -only PASS: the password is used as is regular credentials
		JUMPSERVER_IP=
		JUMPSERVER_PORT=
		JUMPSERVER_USER=
		JUMPSERVER_PASS=
		SSH_KEYFILE=
		
		#paths
		CONFIGPATH=/var/elbrus/shared/ssh-configs
		MAINPATH=/var/elbrus/ssh-manager/src/main.py
		
		Do you want to run the setup script? (y/n/exit) y
		...
		Initialized empty Git repository in /var/elbrus/shared/ssh-configs/.git/
		info: created config folder!
		Install dependencies ...
		...
		Cleaning up...
		elbrus@server:/ssh-manager$
	\end{lstlisting}
	\newpage
	
	\lstset{style=files}
	\begin{lstlisting}[caption={Ergänzen der fehlenden Werten in '.env'.}, language=bash]
		#values regarding the jumpserver:
		#IP, PORT and USER values must be set!
		#depending on the usage you can set either:
		#   -PASS and KEYFILE: the keyfile is used, the pass is interpreted as the passphrase
		#   -only KEYFILE: the keyfile is used
		#   -only PASS: the password is used as is regular credentials
		JUMPSERVER_IP=
		JUMPSERVER_PORT=
		JUMPSERVER_USER=
		JUMPSERVER_PASS=
		SSH_KEYFILE=
	\end{lstlisting}
	
	\subsubsection{2 - Ohne Setup Script}
	
	\lstset{style=files}
	\begin{lstlisting}[caption={Anhand von '.env.example' eigene '.env' Datei anlegen.}, language=bash]
		#global
		SHAREDCONFIG=/var/elbrus/shared/.config
		
		#values regarding the jumpserver:
		#IP, PORT and USER values must be set!
		#depending on the usage you can set either:
		#   -PASS and KEYFILE: the keyfile is used, the pass is interpreted as the passphrase
		#   -only KEYFILE: the keyfile is used
		#   -only PASS: the password is used as is regular credentials
		JUMPSERVER_IP=
		JUMPSERVER_PORT=
		JUMPSERVER_USER=
		JUMPSERVER_PASS=
		SSH_KEYFILE=
		
		#paths
		CONFIGPATH=/var/elbrus/shared/ssh-configs
		MAINPATH=/var/elbrus/ssh-manager/src/main.py
	\end{lstlisting}
	
	\lstset{style=commands}	
	\begin{lstlisting}[caption={Ausführen des Scripts zur Initialisierung des VCS Verzeichnisses.}]
		elbrus@server:/var/elbrus/ssh-manager$ bash src/setup.sh
	\end{lstlisting}
	
	\lstset{style=commands}
	\begin{lstlisting}[caption={Installieren von fehlenden python3 Packages.}]
		elbrus@server:/var/elbrus/ssh-manager$ pip3 install -r requirements.txt
	\end{lstlisting}
	\newpage
	
	\subsection[systemd service]{Der Systemd Service}
	
	\lstset{style=files}
	\begin{lstlisting}[caption={ssh-manager.service.example - Die Variable 'WorkingDirectory' sowie die Variable 'User' anpassen.},language=bash ,keywords={WorkingDirectory, User}, keywordstyle=\color{red}, firstnumber=5]
		...
		#job is starting immediatly after the start action has been called
		Type=simple
		#the user to execute the script
		User=elbrus
		#the working directory
		WorkingDirectory=/var/elbrus/ssh-manager/src/
		#which script should be executed
		ExecStart=/bin/bash routine.sh
		...
	\end{lstlisting}
	
	\lstset{style=commands}
	\begin{lstlisting}[caption={Kopieren des Serviceprogrammes.}]
		elbrus@server:/var/elbrus$ sudo cp ssh-manager/src/ssh-manager.service.example \
		/etc/systemd/system/ssh-manager.service
	\end{lstlisting}
	
	\lstset{style=commands}
	\begin{lstlisting}[caption={Kopieren des Zeitplanungsprogrammes.}]
		elbrus@server:/var/elbrus$ sudo cp ssh-manager/src/ssh-manager-schedule.timer.example \
		/etc/systemd/system/ssh-manager-schedule.timer
	\end{lstlisting}
	
	\lstset{style=commands}
	\begin{lstlisting}[caption={Neuladen des 'systemctl' Deamons.}]
		elbrus@server:/var/elbrus$ sudo systemctl daemon-reload
	\end{lstlisting}
	
	\lstset{style=commands}
	\begin{lstlisting}[caption={Aktivieren des Serviceprogrammes.}]
		elbrus@server:/var/elbrus$ sudo systemctl enable ssh-manager.service
	\end{lstlisting}
	
	\lstset{style=commands}
	\begin{lstlisting}[caption={Aktivieren des Zeitplanungsprogrammes.}]
		elbrus@server:/var/elbrus$ sudo systemctl enable ssh-manager-schedule.timer
	\end{lstlisting}
	
	\lstset{style=commands}
	\begin{lstlisting}[caption={Starten des Zeitplanungsprogrammes.}]
		elbrus@server:/var/elbrus$ sudo systemctl start ssh-manager-schedule.timer
	\end{lstlisting}