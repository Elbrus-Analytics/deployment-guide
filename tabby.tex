% --------- deployment guide for tabby ---------		
	\subsection{1 - Mit Setup Script}
	
	\lstset{style=commands}
	\begin{lstlisting}[caption={Ausführen des 'install.sh' Scripts.}]
		elbrus@server:~$ cd /var/elbrus
		elbrus@server:/var/elbrus$ sudo bash tabby/install.sh
		Do you want to proceed with setup of the 'tabby'? (y/n) y
		
		Where is the shared config stored [/var/elbrus/shared/.config]:
		Where should the tabby be stored [/var/elbrus/tabby]:
		Where should the log be stored (dir) [/var/elbrus/shared/log]:
		Where should the traces be stored (dir) [/var/elbrus/shared/traces]:
		
		Is the shared config stored at '/var/elbrus/shared/.config' ?
		Should the capture-device be stored at '/var/elbrus/tabby' ?
		Should the log be stored at '/var/elbrus/shared/log' ?
		Should the traces be stored at '/var/elbrus/shared/traces' ? (y/n/exit) y
		
		info: a listing of your interfaces follows
		
		...
		
		Which inteface should be captured (name)? eth0if81
		
		Should 'eth0if81' be captured? (y/n/exit) y
		info: the capturing interface can be changed in the environment file.
		
		...
		
		<@\textcolor{codegreen}{Compiling}@> proc-macro2 v1.0.38
		<@\textcolor{codegreen}{Compiling}@> unicode-xid v0.2.3
		<@\textcolor{codegreen}{Compiling}@> syn v1.0.93
		<@\textcolor{codegreen}{Compiling}@> libpcap-tools v0.1.0
		<@\textcolor{codegreen}{Compiling}@> tokio-postgres v0.7.6
		<@\textcolor{codegreen}{Compiling}@> pcap-importer v0.1.0
		
		...
		
		elbrus@server:/var/elbrus$
	\end{lstlisting}
	\newpage
	
	\subsection{2 - Ohne Setup Script}
	\subsubsection[file config]{Umgebung Konfigurieren}
	\lstset{style=files}
	\begin{lstlisting}[caption={Anhand von '.env.example' eigene '.env' Datei anlegen.}, language=bash]
		#global
		SHAREDCONFIG=/var/elbrus/shared/.config
		
		#paths
		PCAPFOLDER=/var/elbrus/shared/traces
		LOGFILEDIR=/var/elbrus/shared/log
		CAPTUREPATH=/var/elbrus/tabby/tabby
		
		#settings
		TIMEPERCAPTURE=600
		TIMEPERIMPORT=300
		TIMEPERREPORT=900
		MAXFILES=12
		INTERFACE=eth0if81
	\end{lstlisting}
	
	\lstset{style=commands}	
	\begin{lstlisting}[caption={Kompilieren der Kernsoftware.}]
		elbrus@server:/var/elbrus$ sudo /root/.cargo/bin/cargo build --release \
		--manifest-path="/var/elbrus/tabby/Cargo.toml"
	\end{lstlisting}
	
	\lstset{style=commands}
	\begin{lstlisting}[caption={Verschieben der kompilierten Software.}]
		elbrus@server:/var/elbrus$ mv tabby/target/release/tabby tabby/tabby
	\end{lstlisting}

	\lstset{style=commands}
	\begin{lstlisting}[caption={Verschieben der kompilierten Software.}]
		elbrus@server:/var/elbrus$ mv tabby/target/release/tabby tabby/tabby
	\end{lstlisting}

	\lstset{style=commands}
	\begin{lstlisting}[caption={Anlegen des Benutzers 'tabby'.}]
		elbrus@server:/var/elbrus$ sudo useradd -G elbrus tabby
	\end{lstlisting}

	\lstset{style=commands}
	\begin{lstlisting}[caption={Verändern der berechtigungen auf Ordner 'tabby'.}]
		elbrus@server:/var/elbrus$ sudo chown -R tabby:tabby tabby
	\end{lstlisting}
	\newpage

	\lstset{style=commands}
	\begin{lstlisting}[caption={Anlegen des Log-Verzeichnisses.}]
		elbrus@server:/var/elbrus$ mkdir -p /var/elbrus/shared/log
	\end{lstlisting}

	\lstset{style=commands}
	\begin{lstlisting}[caption={Verändern der Berechtigung das Log-Verzeichnisses.}]
		elbrus@server:/var/elbrus$ chmod 777 -R /var/elbrus/shared/log
	\end{lstlisting}

	\lstset{style=commands}
	\begin{lstlisting}[caption={Verändern der Berechtigung der kompilierten Software.}]
		elbrus@server:/var/elbrus$ sudo chmod 750 tabby/tabby
		elbrus@server:/var/elbrus$ sudo setcap cap_net_raw,cap_net_admin=eip tabby/tabby
	\end{lstlisting}

	\subsubsection[systemd service]{Der Systemd Service}
	
	\lstset{style=files}
	\begin{lstlisting}[caption={tabby.service.example - Die Variable 'WorkingDirectory' sowie die Variable 'User' anpassen.},language=bash ,keywords={WorkingDirectory, User}, keywordstyle=\color{red}, firstnumber=5]
		...
		#job is starting immediatly after the start action has been called
		Type=simple
		#the user to execute the script
		User=tabby
		#the working directory
		WorkingDirectory=/var/elbrus/tabby/
		#which script should be executed
		ExecStart=/bin/bash tabby.sh
		...
	\end{lstlisting}

	\lstset{style=files}
	\begin{lstlisting}[caption={tabby-error-handler.service.example - Die Variable 'WorkingDirectory' sowie die Variable 'User' anpassen.},language=bash ,keywords={WorkingDirectory, User}, keywordstyle=\color{red}, firstnumber=5]
		...
		[Service]
		Type=oneshot
		User=tabby
		WorkingDirectory=/var/elbrus/tabby/
		ExecStart=/bin/bash tabby-log.sh
		...
	\end{lstlisting}
	\newpage
	
	\lstset{style=commands}
	\begin{lstlisting}[caption={Kopieren des Serviceprogrammes.}]
		elbrus@server:/var/elbrus$ sudo cp tabby/tabby.service.example \
		/etc/systemd/system/tabby.service
	\end{lstlisting}
	
	\lstset{style=commands}
	\begin{lstlisting}[caption={Kopieren des Errorhandlers.}]
		elbrus@server:/var/elbrus$ sudo cp tabby/tabby-error-handler.service.example \
		/etc/systemd/system/tabby-error-handler.service
	\end{lstlisting}
	
	\lstset{style=commands}
	\begin{lstlisting}[caption={Neuladen des 'systemctl' Deamons.}]
		elbrus@server:/var/elbrus$ sudo systemctl daemon-reload
	\end{lstlisting}
	
	\lstset{style=commands}
	\begin{lstlisting}[caption={Aktivieren des Serviceprogrammes.}]
		elbrus@server:/var/elbrus$ sudo systemctl enable tabby.service
	\end{lstlisting}
	
	\lstset{style=commands}
	\begin{lstlisting}[caption={Starten des Serviceprogrammes.}]
		elbrus@server:/var/elbrus$ sudo systemctl start tabby.service
	\end{lstlisting}