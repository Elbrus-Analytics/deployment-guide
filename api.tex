% --------- deployment guide for the API ---------		
	\subsection{1 - Mit Setup Script}
	
	\lstset{style=commands}
	\begin{lstlisting}[caption={Ausführen des 'install.sh' Scripts.}]
		elbrus@server:~$ cd /var/elbrus
		elbrus@server:/var/elbrus$ sudo bash api/install.sh
		Do you want to proceed with the setup of the 'api'? (y/n) y
		
		Where should the 'api' be stored [/var/elbrus/api]:
		Where is the shared config stored [/var/elbrus/shared/.config]:
		
		Is the shared config stored at '/var/elbrus/shared/.config'?
		Do you want to store the 'api' at '/var/elbrus/api'? (y/n/exit) y
		
		info: installing dependencies
		
		...
		
		On which Port should the api run ['3000']? 3000
		On which URL is the webinterface running? (e.g. http://1.2.3.4:80/) http://1.2.3.4:80/
		
		Is your webinterface running on 'http://1.2.3.4:80/'?
		Should the api run on port '3000'? (y/n/exit) y
		
		info: Be aware if you skip you will have to enter the values manually in 
		'/var/elbrus/api/.env' and run 'pm2 restart elb-api'
		Do you want to configure the api email settings (y/skip/exit): y
		What is your SMTP Host (e.g. smtp.gmail.com): smtp.gmail.com
		What is your SMTP Port (e.g. 465): 465
		What is your SMTP Username (e.g. elbrus): elbrus
		What is your SMTP Password (e.g. ******): elbrus
		Who should send the email (e.g. info@gmail.com): info@gmail.com
		Which sender should be displayed (e.g ELBRUS SYSTEM): ELBRUS SYSTEM
		Please check the following settings:
		HOST: smtp.gmail.com
		PORT: 465
		USER: elbrus
		PASSWORD: elbrus
		EMAIL: info@gmail.com
		DISPLAYNAME: ELBRUS SYSTEM
		are these your settings? (y/n/exit) y
		
		<@\textcolor{codegreen}{Success!}@> .env file was created
		
		...
		
		elbrus@server:/var/elbrus$
	\end{lstlisting}
	\newpage
	
	\subsection{2 - Ohne Setup Script}
	\subsection{Voraussetzungen}
	
	\lstset{style=commands}
	\begin{lstlisting}[caption={Nachinstallieren der Abhängigkeiten.}]
		elbrus@server:/var/elbrus$ cd api
		elbrus@server:/var/elbrus/api$ sudo npm install
	\end{lstlisting}

	\subsection[file config]{Umgebung Konfigurieren}
	
	\lstset{style=files}
	\begin{lstlisting}[caption={Anhand von '.env.example' eigene '.env' Datei anlegen.}, language=bash]
		# Application Name
		APP_NAME=Elbrus-API
		
		# Port number
		PORT=3000
		
		# BASE URL
		BASE=https://localhost:3000
		
		# Path to the shared config
		SHARED_CONFIG=/var/elbrus/shared/config.txt
		
		# JWT
		JWT_SECRET=thisisasamplesecret
		JWT_ACCESS_EXPIRATION_MINUTES=30
		JWT_REFRESH_EXPIRATION_DAYS=30
		
		# SMTP configuration options for the email service
		SMTP_HOST=
		SMTP_PORT=
		SMTP_USERNAME=
		SMTP_PASSWORD=
		EMAIL_FROM=
		EMAIL_NAME=
		
		# Path where ssh-manager stores configs archive
		VCS_Path=/var/elbrus/shared
	\end{lstlisting}
	\newpage
	
	\begin{enumerate}
		\item \textbf{APP\_NAME} wird rein als beschreibender Name genutzt und kann so belassen werden.
		\item \textbf{PORT} beschreibt den TCP Port auf dem die Applikation laufen soll.
		\item \textbf{BASE} ist der Wert der Basis URL auf welche zugegriffen wird. Hier muss der Port auch angegeben werden!
		\item \textbf{DB\_USER} ist der benutzername des DBMS Benutzers, über welchen der Zugriff auf die Datenbank läuft.
		\item \textbf{DB\_HOST} ist der hostname/ip-adresse des Servers welcher die Datenbank hostet.
		\item \textbf{DB\_DATABASE} beschreibt den Namen der Datenbank selber.
		\item \textbf{DB\_PASSWORD} ist das Passwort des DBMS Benutzers, über welchen der Zugriff auf die Datenbank läuft.
		\item \textbf{DB\_PORT} ist der TCP Port des Servers welcher die Datenbank hostet.
		\item \textbf{JWT\_SECRET} ist das Passwort mit dem alle JWT Tokens ausgestellt werden.
		\item \textbf{JWT\_ACCESS\_EXPIRATION\_MINUTES} gibt die Dauer der Gültigkeit eines Access-Tokens an (in Minuten)
		\item \textbf{JWT\_REFRESH\_EXPIRATION\_DAYS} gibt die Dauer der Gültigkeit eines Refresh-Tokens an (in Tagen)
		\item \textbf{SMTP\_HOST} ist der hostname/ip-adresse des EMail Servers
		\item \textbf{SMTP\_PORT} ist der TCP Port des EMail Servers für SMTP
		\item \textbf{SMTP\_USERNAME} ist der username des Benutzers zum einloggen in den EMail Account
		\item \textbf{SMTP\_PASSWORD} ist das passwort des Benutzers zum einloggen in den EMail Account
		\item \textbf{EMAIL\_FROM} gibt die Email adresse an, von welcher gesendet werden soll.
		\item \textbf{EMAIL\_NAME} gibt den Namen an, welcher dem Empfänger angezeigt werden soll.
	\end{enumerate}

	\subsection{Inbetriebnahme}
	
	\lstset{style=commands}
	\begin{lstlisting}[caption={Starten der API.}]
		elbrus@server:/var/elbrus/api$ pm2 start ecosystem.config.json
	\end{lstlisting}