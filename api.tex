% --------- deployment guide for the API ---------		
	\subsection{Voraussetzungen}
	
	\lstset{style=commands}
	\begin{lstlisting}[caption={Installieren von 'pm2'.}]
		elbrus@server:~/var/elbrus$ sudo npm install -g pm2
	\end{lstlisting}
	
	\lstset{style=commands}
	\begin{lstlisting}[caption={Nachinstallieren der Abhängigkeiten.}]
		elbrus@server:~/var/elbrus$ cd api
		elbrus@server:~/var/elbrus/api$ sudo npm install
	\end{lstlisting}
	\newpage
	
	\subsection[file config]{Umgebung Konfigurieren}
	
	\lstset{style=files}
	\begin{lstlisting}[caption={Anhand von '.env.example' eigene '.env' Datei anlegen.}, language=bash]
		# Application Name
		APP_NAME=Elbrus-API
		
		# Port number
		PORT=3000
		
		# BASE URL
		BASE=https://localhost:3000
		
		# URL of DB
		DB_USER=
		DB_HOST=
		DB_DATABASE=
		DB_PASSWORD=
		DB_PORT=
		
		# JWT
		JWT_SECRET=thisisasamplesecret
		JWT_ACCESS_EXPIRATION_MINUTES=30
		JWT_REFRESH_EXPIRATION_DAYS=30
		
		# SMTP configuration options for the email service
		SMTP_HOST=
		SMTP_PORT=
		SMTP_USERNAME=
		SMTP_PASSWORD=
		EMAIL_FROM=
		EMAIL_NAME=
	\end{lstlisting}
	
	\subsection{Inbetriebnahme}
	
	\lstset{style=commands}
	\begin{lstlisting}[caption={Starten der API.}]
		elbrus@server:~/var/elbrus/api$ pm2 start ecosystem.config.json
	\end{lstlisting}
	Die API läuft in folge automatisch im Hintergrund.
	\newpage
	
	\begin{enumerate}
		\item \textbf{APP\_NAME} wird rein als beschreibender Name genutzt und kann so belassen werden.
		\item \textbf{PORT} beschreibt den TCP Port auf dem die Applikation laufen soll.
		\item \textbf{BASE} ist der Wert der Basis URL auf welche zugegriffen wird. Hier muss der Port auch angegeben werden!
		\item \textbf{DB\_USER} ist der benutzername des DBMS Benutzers, über welchen der Zugriff auf die Datenbank läuft.
		\item \textbf{DB\_HOST} ist der hostname/ip-adresse des Servers welcher die Datenbank hostet.
		\item \textbf{DB\_DATABASE} beschreibt den Namen der Datenbank selber.
		\item \textbf{DB\_PASSWORD} ist das Passwort des DBMS Benutzers, über welchen der Zugriff auf die Datenbank läuft.
		\item \textbf{DB\_PORT} ist der TCP Port des Servers welcher die Datenbank hostet.
		\item \textbf{JWT\_SECRET} ist das Passwort mit dem alle JWT Tokens ausgestellt werden.
		\item \textbf{JWT\_ACCESS\_EXPIRATION\_MINUTES} gibt die Dauer der Gültigkeit eines Access-Tokens an (in Minuten)
		\item \textbf{JWT\_REFRESH\_EXPIRATION\_DAYS} gibt die Dauer der Gültigkeit eines Refresh-Tokens an (in Tagen)
		\item \textbf{SMTP\_HOST} ist der hostname/ip-adresse des EMail Servers
		\item \textbf{SMTP\_PORT} ist der TCP Port des EMail Servers für SMTP
		\item \textbf{SMTP\_USERNAME} ist der username des Benutzers zum einloggen in den EMail Account
		\item \textbf{SMTP\_PASSWORD} ist das passwort des Benutzers zum einloggen in den EMail Account
		\item \textbf{EMAIL\_FROM} gibt die Email adresse an, von welcher gesendet werden soll.
		\item \textbf{EMAIL\_NAME} gibt den Namen an, welcher dem Empfänger angezeigt werden soll.
	\end{enumerate}